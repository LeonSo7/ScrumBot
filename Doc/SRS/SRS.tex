\documentclass[12pt, titlepage]{article}

\usepackage{booktabs}
\usepackage{tabularx}
\usepackage{hyperref}
\usepackage{setspace}
\usepackage{xspace}
\usepackage{float}
\usepackage{placeins}
\usepackage{enumerate}
\usepackage[shortlabels]{enumitem}
\usepackage{array}
\usepackage{longtable}

% Document margins
\usepackage[left=1in,top=1in,right=1in,bottom=1in]{geometry}

\newcolumntype{L}[1]{>{\raggedright\let\newline\\\arraybackslash\hspace{0pt}}m{#1}}

\hypersetup{
    colorlinks,
    citecolor=black,
    filecolor=black,
    linkcolor=red,
    urlcolor=blue
}
\usepackage[round]{natbib}

\title{SE 3XA3: Software Requirements Specification\\ScrumBot}

\author{
    Team 304, ScrumBot
        \\ Arkin Modi, modia1
        \\ Leon So, sol4
        \\ Timothy Choy, choyt2
}
\date{Last Updated: February 9, 2020}

%\input{../Comments}

\begin{document}

\maketitle

\pagenumbering{roman}
\tableofcontents
\listoftables
\listoffigures

\newpage

\begin{table}[!hbp]
    \caption{Revision History} \label{TblRevisionHistory}
    \begin{tabularx}{\textwidth}{llX}
        \toprule
            \textbf{Date} & \textbf{Developer(s)} & \textbf{Change}\\
        \midrule
            January 23, 2020 & Arkin Modi & Copy template\\
            February 3, 2020 & Leon So & Worked on Stakeholders section\\
            February 4, 2020 & Leon So & Worked on Purpose, Non-functional Requirements sections\\
            February 8, 2020 & Leon So & Formatted and added non-functional requirements; Created terminology table; worked on Project Scope; Relevant Facts and Assumptions\\
            February 8, 2020 & Arkin Modi & Worked on Mandated Constraints, Off-the-Shelf Solutions, New Problems, Installability Requirement\\
            February 8, 2020 & Timothy Choy & Worked on Look and Feel Requirements, Legal Requirements\\
            February 8, 2020 & Timothy Choy & Updated SRS formatting, Functional Requirements\\
            February 8, 2020 & Leon So & Add Functional Requirements, update Stakeholders section, worked on Project Issues Section\\
            February 8, 2020 & Timothy Choy & Worked on Project Planning, Planning of the Development Phases, and Non-functional Requirements sections\\
            February 9, 2020 & Timothy Choy & Worked on Work Partitioning, Individual Product Use Cases\\
            February 9, 2020 & Everyone & Final Revision\\
        \bottomrule
    \end{tabularx}
\end{table}

\FloatBarrier

\newpage

\pagenumbering{arabic}

This document describes the requirements for ScrumBot. The template for the Software Requirements Specification (SRS) is a subset of the Volere template~\citep{RobertsonAndRobertson2012}.
% If you make further modifications to the template, you should explicitly state what modifications were made.

\section{Project Drivers}

\subsection{The Purpose of the Project}
Scrum is an agile process framework widely used in industry for managing and coordinating collaborative projects. Scrum being a process based on the agile development method, follows a highly iterative process and often has heavy customer involvement, therefore it can be often be complex. It is often difficult to manage communication and coordination within the project due to the complexity of agile development processes. With Discord being a popular communication tool used by many teams of software developers today, Scrumbot provides a solution that directly integrates the management of a scrum development cycle into the communication channels. Scrumbot will allow for better management and organization of retrospectives, stand-ups, and other scrum/agile stages used by software teams within their routine communication channel.\\

As a result, Scrumbot, a well-integrated scrum management system, can reduce inefficiencies surrounding the management of projects using the scrum framework, as well as strengthen communication within the agile process. This will help software development firms reduce costs and achieve better workflow.

\subsection{The Stakeholders}
This solution will be used in an environment where the development team uses the scrum agile framework and uses Discord as their routine communication channel.

\subsubsection{The Client}
\begin{itemize}
    \item Dr. Ashgar Bokhari (3XA3)
    \begin{itemize}
      \item[] The client for this project is Dr. Ashgar Bokhari - professor of the SFWRENG 3XA3 course.
    \end{itemize}
\end{itemize}

\subsubsection{The Customer}
The customer of the ScrumBot is any firm looking to improve efficiency in communication and organization during the Scrum process of an agile software development cycle using the Scrum framework. This customer will likely be a software development firm (or the development department of a firm).

\begin{itemize}
    \item Software Development Firm
    \begin{itemize}
      \item[] The development firm will benefit through centralizing project resources, therefore improving the organization and management of the project. The development firm will therefore be able to reduce any costs associated with inefficiencies during the development process. This will also improve communication between members of the development firm.
    \end{itemize}
\end{itemize}
\subsubsection{The Users (Roles in Scrum process)}
\begin{itemize}
    \item Product Owner(s)
    \begin{itemize}
      \item[] The product owner(s) will benefit from the reduced development time through improved communication and efficient management of the project. The product owner(s) will also benefit as their ideas and requirements will be better organized, thus helping the development team more efficiently achieve the goals of the product owner(s).
    \end{itemize}
    \item Scrum Master
    \begin{itemize}
      \item[] The scrum master will be better able to coordinate scrum plans and division of tasks, as well as routine communication (i.e. retrospectives and stand-ups). With a scrum management system directly implemented into the communication channels, the scrum master will be able to directly manage the meeting information and their scrum development cycle during communication. (Note that Scrum Master is also a member of the Development Team).
    \end{itemize}
    \item Development Team
    \begin{itemize}
      \item[] The development team will be able to more easily follow through on tasks assigned by the scrum master. The development team may easily refer to any relevant articles or information regarding their project. This will reduce inefficiencies and speedup communication within the development team. It will also reduce confusion, as all information concerning the development cycle will be available through their routine communication channel (Discord).
    \end{itemize}
    \item Business Analyst
    \begin{itemize}
      \item[] The Business Analyst will be able to better coordinate communication concerning the project between the development team and the product owner(s). ScrumBot will also allow the Business Analyst to better: organize and monitor the progress of the project, update requirements, and analyze sprint progress.
    \end{itemize}
\end{itemize}
% \subsubsection{Other stakeholders}


\subsection{Mandated Constraints}
\subsubsection{Solution Constraints}
\noindent \textit{Description:} The project shall be written in the Python 3 programming language.\\
\textit{Rationale:} All developers are familiar with Python 3 and there is an existing API available in Python 3.\\
\textit{Fit Criterion:} The software is written in Python 3.\\

\noindent \textit{Description:} The application shall function within Discord.\\
\textit{Rationale:} Discord is a popular communication tool used by many software development team. In addition, the existing project utilizes Discord to perform all of its features.\\
\textit{Fit Criterion:} All application features are fully operational within Discord.

\subsubsection{Implementation Environment of the Current System}
The application will be be installed on the user's local machine or a server, and added to a Discord server. To function, the application be started on the machine and the user will interact with it through the respective Discord server.

\subsubsection{Partner or Collaborative Applications}
The application will use Discord and the Discord API to function as an interface.

\subsubsection{Off-the-Shelf Software}
The following off-the-shelf software will utilized:
\begin{itemize}
    \item Discord (Available at \url{https://discordapp.com/})
    \item Discord Server (Available through Discord)
    \item Python 3 (Available at \url{https://www.python.org/downloads/})
    \item discord.py (Available at \url{https://pypi.org/project/discord.py/})
\end{itemize}
All off-the-shelf software are available for free.

\subsubsection{Anticipated Workplace Environment}
The anticipated workplace environment for this application is within a software development team's Discord server. This application can be used from anywhere, as long as the user has access to the respective Discord server.

\subsubsection{Schedule Constraints}
The project deliverable must completed by their respective deadline. The remaining deadline include:
\begin{itemize}
    \item Proof of Concept (February 13, 2020, 1:00 PM)
    \item Test Plan, Revision 0 (February 28, 2020, 11:30 PM)
    \item Design \& Document, Revision 0 (March 13, 2020, 11:30 PM)
    \item Demonstration, Revision 0 (March 17, 2020)
    \item Demonstration, Revision 1 (March 31, 2020)
    \item Final Documentation, Revision 1 (April 6, 2020, 11:30 PM)
\end{itemize}

\subsubsection{Budget Constraints}
This project has no monetary budget. If there are any necessary purchases for development, the cost shall be paid by the project members. All resources to re-create and upgrade the existing project are provided.

\subsubsection{Enterprise Constraints}
This application will be available for free to any user that has access to a Discord server and a machine with Python 3 (and the dependencies) installed.

\subsection{Naming Conventions and Terminology}
\begin{table}[H]
    \caption{Naming Conventions and Terminology}
    \centering
    \begin{tabular}{ |p{5cm}|p{10.5cm}|  }
     \hline
     \textbf{Term} & \textbf{Definition} \\
     \hline
     ScrumBot & The name of the Discord Bot \\
     \hline
     Discord & A cross-platform chat application\\
     \hline
     Trello & A web-based Kanban-style list-making application\\
     \hline
     Discord Bot & An automated chat bot that operates on Discord\\
     \hline
     Agile Development Method & A software development methodology\\
     \hline
     Scrum & An Agile process framework\\
     \hline
     Scrum Master & The facilitator for an agile development team who plans, leads and organizes Scrum meetings\\
     \hline
     Stand-up & A daily coordination meeting used in the Scrum framework\\
     \hline
     Sprint & A set time period where specific work has to be completed and made ready for review\\
     \hline
     Retrospective & A team meeting for reflecting on an Scrum sprint\\
     \hline
     Business Analyst & Communicates and coordinates project requirements and deadlines between the Product Owner(s), Scrum Master, and Development Team\\
     \hline
     Grooming & A meeting where the Business Analyst communicates and coordinates project requirements and deadlines with the Scrum Master and Development Team\\
     \hline
     BE & Business Event\\
     \hline
    \end{tabular}
\end{table}

\FloatBarrier

\subsection{Relevant Facts and Assumptions}

It is assumed that the user will be deploying the Discord bot on Discord. It is assumed that the user is familiar with the Scrum Agile process framework and the Agile development software design methodology.

\section{Functional Requirements}

\subsection{The Scope of the Work and the Product}
ScrumBot is a specialized chat bot (i.e. a Discord Bot) that runs on the Discord application. ScrumBot will allow users to better manage software development projects using the Scrum Agile Framework. The ScrumBot will have features which help software development teams organize and record Scrum meeting information, as well as features which will help software development teams stay on track. This will allow better organization of Scrum sprints, meetings  (i.e. sprint-planning, retrospectives  and  stand-ups) and  improve  efficiency  within  the communication channel.

\subsubsection{The Context of the Work}
This product is designed to be used by software development teams using: Discord as their main communication channel, an Agile development software design methodology, and the Scrum agile process framework. Additionally, this application can be integrated with Trello, and Google Calendar and Maps.

\subsubsection{Work Partitioning}
\begin{longtable}{|L{4cm}|L{3cm}|L{1.5cm}|L{6cm}|}
    \caption{Work Partitioning}
    \label{tab:my_label}\\
    \hline
    Event Name & Input & Output & Summary \\
    \hline
    Scrumbot Initialization & Development Firm & Discord channel & Adding the Discord bot to a Discord server for the first time\\
    \hline
    Project Creation & Business Analyst & & Adding a new project to Scrumbot\\
    \hline
    Project Removal & Business Analyst & & Removing a project from Scrumbot\\
    \hline
    Sprint-planning Meeting Occurs & Development Team & Discord channel & A meeting is held where a sprint is planned, held prior to the sprint\\
    \hline
    Stand-up Meeting Occurs & Development Team & Discord channel & A meeting is held, usually daily, where members record what they have done\\
    \hline
    Retrospective Meeting Occurs & Development Team & Discord channel & A meeting is held where feedback is given on the previous sprint, held after each sprint\\
    \hline
    Grooming Meeting Occurs & Business Analyst \& Development Team & Discord channel & A meeting in which the requirements of the project are specified by the Business Analyst, and communicated to the Development Team\\
    \hline
    Meeting Creation & Any Discord member & Discord channel & Creates a scheduled meeting into the Discord channel\\
    \hline
    Meeting Removal & Any Discord member & Discord channel & Removes a scheduled meeting from the Discord channel\\
    \hline
    List Scheduled Meetings & Any Discord member & Discord channel & A command that lists all the scheduled meetings of a given member\\
    \hline
    List Tasks & Any Discord member & Discord & A command that lists all the scheduled tasks for a given member\\
    \hline
\end{longtable}

\subsubsection{Individual Product Use Cases}
This product is primarily used by software development firms, from start-ups to large companies, as these companies will be the most likely users of the Scrum agile framework. They would enjoy the benefits of using Discord as a means for quick communication, with integrated abilities to work on projects using Scrum.\\

Another use case could be students who want to plan projects and work on them through the Scrum method. A lot of students would benefit from this over other forms of communication, such as Slack, a large portion of students already have access to Discord.

\subsection{Functional Requirements}
\begin{enumerate}[{BE}1.]
    \item The development firm wants to add ScrumBot to its Discord channel
    \begin{enumerate}[{VP}1.]
        \item Viewpoint: Development Firm
            \begin{enumerate}
                \item The system shall ask for administrative permissions to the Discord channel
                \item The system shall provide an introduction and basic commands list in the channel
                \item The system shall provide a link to access Scrumbot's documentation
            \end{enumerate}
        \item Viewpoint: Development Team
            \begin{enumerate}
                \item The system shall notify the development team that ScrumBot has been added to the channel
                \item The system shall provide an introduction and basic commands list in the channel
                \item The system shall provide a link to access Scrumbot's documentation
            \end{enumerate}
            
        \item Viewpoint: Business Analyst
            \begin{enumerate}
                \item The system shall notify the Business Analyst that ScrumBot has been added to the channel
                \item The system shall provide an introduction and basic commands list in the channel
                \item The system shall provide a link to access Scrumbot's documentation
            \end{enumerate}
            
        \item Viewpoint: Scrum Master
            \begin{enumerate}
                \item The system shall notify the Scrum Master that ScrumBot has been added to the channel
                \item The system shall provide an introduction and basic commands list in the channel
                \item The system shall provide a link to access Scrumbot's documentation
            \end{enumerate}
    \end{enumerate}

	\item The Business Analyst wants to add a new project
	\begin{enumerate}[{VP}1.] 
	    \item Viewpoint: Development Firm
	        \begin{enumerate}
	            \item The system shall add the project for all roles and personnel involved with the project
	        \end{enumerate}
	    \item Viewpoint: Development Team
	        \begin{enumerate}
	            \item The system shall update the project list of the development team
	            \item The system shall notify the development team of the new project
	            \item The system shall notify the development team of the scheduled first meeting
	        \end{enumerate}
		\item Viewpoint: Business Analyst
			\begin{enumerate}
			    \item The system shall prompt the Business Analyst for the project details (name, description, product owner)
			    \item The system shall prompt the Business Analyst for first meeting details
			    \item The system shall append the project and its details to the project list
			\end{enumerate}
		\item Viewpoint: Scrum Master
			\begin{enumerate}
			    \item The system shall update the project list of the Scrum Master
			    \item The system shall notify the Scrum Master of the new project
			    \item The system shall notify the Scrum Master of the scheduled first meeting
			\end{enumerate}
	\end{enumerate}

	\item The Business Analyst wants to remove a project
	\begin{enumerate}[{VP}1.] 
	    \item Viewpoint: Development Firm
	        \begin{enumerate}
	            \item The system shall remove the project for all roles and personnel involved with the project
	        \end{enumerate}
	    \item Viewpoint: Development Team
	        \begin{enumerate}
	            \item The system shall update the project list of the development team
	            \item The system shall notify the development team of the removed project
	        \end{enumerate}
		\item Viewpoint: Business Analyst
			\begin{enumerate}
			    \item The system shall ask the Business Analyst for confirmation
			    \item The system shall update the project list of
			    \item The system shall ask the Business Analyst if the project is completed or cancelled
			\end{enumerate}
		\item Viewpoint: Scrum Master
			\begin{enumerate}
	            \item The system shall update the project list of the Scrum Master
	            \item The system shall notify the Scrum Master of the removed project
			\end{enumerate}
	\end{enumerate}
	
	\item A sprint-planning meeting occurs % Meeting where they plan the next sprint, approximately biweekly
	\begin{enumerate}[{VP}1.] 
	    \item Viewpoint: Development Firm
	        \begin{enumerate}
	            \item[] N/A
	        \end{enumerate}
	    \item Viewpoint: Development Team
	        \begin{enumerate}
	            \item The system shall allow the Development Team to see new feedback
	            \item The system shall allow the Development Team to see the updated backlog tasks
	            \item The system shall allow the Development Team to see new goals
	            \item The system shall allow the Development Team to assign tasks to its developers
	            \item The system shall update the Trello Kanban board
	        \end{enumerate}
		\item Viewpoint: Business Analyst
			\begin{enumerate}
			    \item[] N/A
			\end{enumerate}
		\item Viewpoint: Scrum Master
			\begin{enumerate}
	            \item The system shall allow the Scrum Master to add goals for the sprint
	            \item The system shall allow the Scrum Master to add feedback
	            \item The system shall allow the Scrum Master to add backlog tasks
	            \item The system shall allow the Scrum Master to add tasks
	            \item The system shall update the Trello Kanban board
			\end{enumerate}
	\end{enumerate}

	\item A stand-up meeting occurs     % Stand up - Meeting where you talk about what you did, daily
	\begin{enumerate}[{VP}1.] 
	    \item Viewpoint: Development Firm
	        \begin{enumerate}
	            \item[] N/A
	        \end{enumerate}
	    \item Viewpoint: Development Team
	        \begin{enumerate}
	            \item The system shall allow the development team to update and record their progress
	        \end{enumerate}
		\item Viewpoint: Business Analyst
			\begin{enumerate}
			    \item[] N/A
			\end{enumerate}
		\item Viewpoint: Scrum Master
			\begin{enumerate}
			    \item The system shall allow the Scrum Master to update and record their progress
			\end{enumerate}
	\end{enumerate}

    \item A retrospective meeting occurs     % Retrospective - Meeting where you reflect on the sprint, weekly or biweekly
	\begin{enumerate}[{VP}1.] 
	    \item Viewpoint: Development Firm
	        \begin{enumerate}
	            \item[] N/A
	        \end{enumerate}
	    \item Viewpoint: Development Team
	        \begin{enumerate}
	            \item The system shall allow the development team members to see the sprint feedback
	        \end{enumerate}
		\item Viewpoint: Business Analyst
			\begin{enumerate}
			    \item[] N/A
			\end{enumerate}
		\item Viewpoint: Scrum Master
			\begin{enumerate}
			    \item The system shall allow the Scrum Master to add feedback regarding the sprint
			\end{enumerate}
	\end{enumerate}
	
	\item A grooming meeting occurs % Meeting where Business Analyst gives new project details to the team
	\begin{enumerate}[{VP}1.] 
	    \item Viewpoint: Development Firm
	        \begin{enumerate}
	            \item[] N/A
	        \end{enumerate}
	    \item Viewpoint: Development Team
	        \begin{enumerate}
	            \item The system shall allow the development team to see all updates regarding the project requirements, tasks, and deadlines
	        \end{enumerate}
		\item Viewpoint: Business Analyst
			\begin{enumerate}
			    \item The system shall allow the business analyst to update project requirements and deadlines
			\end{enumerate}
		\item Viewpoint: Scrum Master
			\begin{enumerate}
			    \item The system shall allow the Scrum Master to see all updates regarding the project requirements, tasks, and deadlines
			\end{enumerate}
	\end{enumerate}

	\item User wants to add a meeting % Meeting where Business Analyst gives new project details to the team
	\begin{enumerate}[{VP}1.] 
	    \item Viewpoint: Development Firm
	        \begin{enumerate}
	            \item[] N/A
	        \end{enumerate}
	    \item Viewpoint: Development Team
	        \begin{enumerate}
	            \item The system shall notify the invited members of the Development Team
	            \item The system shall prompt the invited members of the Development Team to accept or decline the meeting invitation
	        \end{enumerate}
		\item Viewpoint: Business Analyst
			\begin{enumerate}
	            \item The system shall prompt the user for the meeting subject, time, location, and room
	            \item The system shall prompt the user for the meeting participants
	            \item The system shall invite the participants
	            \item The system shall notify the invited Business Analyst if applicable
	            \item The system shall prompt the invited Business Analyst to accept or decline the meeting invitation, if applicable
			\end{enumerate}
		\item Viewpoint: Scrum Master
			\begin{enumerate}
	            \item The system shall prompt the user for the meeting subject, time, location, and room
	            \item The system shall prompt the user for the meeting participants
	            \item The system shall invite the participants
	            \item The system shall notify the invited Scrum Master if applicable
	            \item The system shall prompt the invited Scrum Master to accept or decline the meeting invitation, if applicable
			\end{enumerate}
	\end{enumerate}
	
	\item User wants to cancel a meeting % Meeting where Business Analyst gives new project details to the team
	\begin{enumerate}[{VP}1.] 
	    \item Viewpoint: Development Firm
	        \begin{enumerate}
	            \item[] N/A
	        \end{enumerate}
	    \item Viewpoint: Development Team
	        \begin{enumerate}
	            \item The system shall notify the participating members from the Development Team
	        \end{enumerate}
		\item Viewpoint: Business Analyst
			\begin{enumerate}
	            \item The system shall notify the Business Analyst if applicable
			\end{enumerate}
		\item Viewpoint: Scrum Master
			\begin{enumerate}
	            \item The system shall notify the Scrum Master if applicable
			\end{enumerate}
	\end{enumerate}
	
	\item User wants to see the scheduled meetings
	\begin{enumerate}[{VP}1.] 
	    \item Viewpoint: Development Firm
	        \begin{enumerate}
	            \item[] N/A
	        \end{enumerate}
	    \item Viewpoint: Development Team
	        \begin{enumerate}
	            \item The system shall display the list of meetings
	        \end{enumerate}
		\item Viewpoint: Business Analyst
			\begin{enumerate}
	            \item The system shall display the list of meetings
			\end{enumerate}
		\item Viewpoint: Scrum Master
			\begin{enumerate}
	            \item The system shall display the list of meetings
			\end{enumerate}
	\end{enumerate}
	
	\item User wants to see the tasks
	\begin{enumerate}[{VP}1.] 
	    \item Viewpoint: Development Firm
	        \begin{enumerate}
	            \item[] N/A
	        \end{enumerate}
	    \item Viewpoint: Development Team
	        \begin{enumerate}
	            \item The system shall display the Trello Kanban board
	        \end{enumerate}
		\item Viewpoint: Business Analyst
			\begin{enumerate}
	            \item[] N/A
			\end{enumerate}
		\item Viewpoint: Scrum Master
			\begin{enumerate}
	            \item The system shall display the Trello kanban board
			\end{enumerate}
	\end{enumerate}
\end{enumerate}


\section{Non-functional Requirements}

\subsection{Look and Feel Requirements}
\subsubsection{Appearance Requirements}
\begin{enumerate}[start=1, label={LF\arabic*.}]
    \item The system shall follow the text format of the Discord application
\end{enumerate}

\subsubsection{Style Requirements}
\begin{enumerate}[start=2, label={LF\arabic*.}]
    \item The system shall use short, clear, and descriptive role names
    \item The system shall colour code users based on their role
\end{enumerate}

\subsection{Usability and Humanity Requirements}
\subsubsection{Ease of Use Requirements}
\begin{enumerate}[start=1, label={UH\arabic*.}]
    \item The system shall have intuitive user commands
    \item The system shall be easy to use for ages 13+
\end{enumerate}
\subsubsection{Personalization and Internationalization Requirements}
\begin{enumerate}[start=3, label={UH\arabic*.}]
    \item The system shall be used by English speakers only
\end{enumerate}
\subsubsection{Learning Requirements}
\begin{enumerate}[start=4, label={UH\arabic*.}]
    \item The system shall have a help menu to explain commands to the user
\end{enumerate}

\subsection{Performance Requirements}
\subsubsection{Speed and Latency Requirements}
\begin{enumerate}[start=1, label={P\arabic*.}]
    \item The system shall respond to user commands within 2 seconds
\end{enumerate}
\subsubsection{Precision or Accuracy Requirements}
\begin{enumerate}[start=2, label={P\arabic*.}]
    \item The system shall have a meeting locations and schedules accuracy greater than 70\%
    \item All meeting times should be accurate to within 30 seconds
\end{enumerate}
\subsubsection{Reliability and Availability Requirements}
\begin{enumerate}[start=4, label={P\arabic*.}]
    \item The system shall load and use the correct data for the team
\end{enumerate}

\subsection{Operational and Environmental Requirements}
\subsubsection{Expected Environment}
\begin{enumerate}[start=1, label={OE\arabic*.}]
    \item The system shall be able to operated and be executed within the Discord application
\end{enumerate}
\subsubsection{Requirements for Interfacing with Adjacent Systems}
\begin{enumerate}[start=2, label={OE\arabic*.}]
    \item The system shall connect to Google's API services
    \item The system shall connect to Discord's API services
    \item The system shall connect to Trello's API services
\end{enumerate}
\subsubsection{Installability Requirement}
\begin{enumerate}[start=5, label={OE\arabic*.}]
    \item It shall be possible for a customer with no special expertise to install the application
    \item The system shall take less than five minutes to install
    \item The system shall be uninstallable
    \item Installation of updates shall not change any setting or user data
\end{enumerate}

\subsection{Maintainability and Support Requirements}
\subsubsection{Maintainability Requirements}
\begin{enumerate}[start=1, label={MS\arabic*.}]
    \item The code shall be documented using comments
    \item The code shall be documented using Doxygen
    \item The code documentation shall be easy to understand
\end{enumerate}
\subsubsection{Supportability Requirements}
\begin{enumerate}[start=4, label={MS\arabic*.}]
    \item The system shall have a help menu available to users at all times
\end{enumerate}
\subsubsection{Longevity Requirements}
\begin{enumerate}[start=5, label={MS\arabic*.}]
	\item The system will be made in modules to increase maintainability and longevity
\end{enumerate}

\subsection{Security Requirements}
\begin{enumerate}[start=1, label={S\arabic*.}]
    \item The connection between the system and the APIs shall use HTTPS for security
\end{enumerate}

\subsection{Cultural Requirements}
\begin{enumerate}[start=1, label={C\arabic*.}]
    \item The system shall use Canadian English spelling
\end{enumerate}

\subsection{Legal Requirements}
\begin{enumerate}[start=1, label={L\arabic*.}]
    \item The system shall not violate any copyrighted properties.
\end{enumerate}

\subsection{Health and Safety Requirements}
% This section is not in the original Volere template, but health and safety are issues that should be considered for every engineering project.
\begin{enumerate}[start=1, label={HS\arabic*.}]
    \item The system shall not harm the user.
\end{enumerate}

\section{Project Issues}
\subsection{Open Issues}
There are currently no open issues.

\subsection{Off-the-Shelf Solutions}
\subsubsection{Ready-Made Products}
\noindent There are open source projects for a Discord Scrum Bot readily available online. Two of which include Scrum Bot by Austen Goddu (\url{https://github.com/Austen-G/Scrum-Bot}) and scrumbot by Colin Brady (\url{https://github.com/colin-brady/scrumbot}).\\

\noindent There are also Scrum Bots available for Slack that should be investigated. Two of which include ScrumGenius (\url{https://scrumgenius.com/}) and Scrum Bot (\url{https://scrumbot.sifts.io/}). ScrumGenius is a paid service.

\subsubsection{Reusable Components}
\noindent The Python library, discord.py, can be used to streamline communications with the Discord server.

\subsubsection{Products That Can Be Copied}
\noindent The existing that this application is a re-make of, can be legally copied and modified.

\subsection{New Problems}
\subsubsection{Effects on the Current Environment}
The application needs to be hosted on a server and unless there is an error or the server gets overloaded, the current environment should not be affected.

\subsubsection{Effects on the Installed Systems}
The application does not interface/coexist with the old implementation.

\subsubsection{Potential User Problems}
Potential user problems as an adverse reaction from interacting with the application during extended use are any problems that would be a result of using a computer. This include but not limited to, carpal tunnel syndrome, computer vision syndrome, and musculoskeletal problems.

\subsubsection{Limitations in the Anticipated Implementation Environment That May Inhibit the New Product}
The server that will host the application will not be powerful enough to handle the amount of network traffic as users are interfacing with it.

\subsubsection{Follow-Up Problems}
In the future, if one or more of the API methods used become deprecated and removed, the application will cease to function.

\subsection{Tasks}
\subsubsection{Project Planning}
The following tasks have been taken from the SFWRENG 3XA3 course outline found on Avenue.
\begin{table}[h]
    \centering
    \caption{Tasks}
    \begin{tabular}{|L{2cm}|L{6cm}|L{6cm}|}
        \hline
        Phase & Task & Due Date \\
        \hline
        Phase 1 & Proof of Concept Demonstration & February 13\\
        \cline{2-3}
         & Test Plan Creation & February 28\\
        \cline{2-3}
         & Design Document & March 13\\
        \cline{2-3}
         & Phase 1 Demonstration & March 16 - 20\\
        \hline
        Phase 2 & Revision to documentation & Throughout March 20 - 30\\
        \cline{2-3}
         & Final Demonstration & March 30 - April 3\\
        \cline{2-3}
         & Final Documentation & April 6\\
        \hline
        
    \end{tabular}
    \label{tab:task_table}
\end{table}

\subsubsection{Planning of the Development Phases}
The development phases are split into two phases:
\begin{enumerate}
    \item Initial development and implementation
    \item Post phase 1 demonstration, revisions to the documentation and implementing the revisions
\end{enumerate}
In phase 1, the structure of our project will be created, including our scope and design ideas. The first phase is important as we will design how every module in our program will interact with one another, and figure out all the pieces necessary to implement the project.\\
The key turning point from phase 1 to phase 2 is the phase 1 demonstration, where our clients will see how the project is implemented and will provide feedback towards revisions of our project.\\
Phase 2 will be about the modifications to our project post-demonstration. Our project will not expect major changes, but instead will be building off of what our clients have given through feedback.

\subsection{Migration to the New Product}
\subsubsection{Requirements for Migration to the New Product}
N/A

\subsubsection{Data That Has to Be Modified or Translated for the New System}
N/A

\subsection{Risks}
There are currently no foreseen risks for this product.

\subsection{Costs}
There are currently no foreseen costs for this project. All software products used are currently free-to-use.

\subsection{User Documentation and Training}
\subsubsection{User Documentation Requirements}
N/A

\subsubsection{Training Requirements}
Basic knowledge in the use of Discord is needed to use this product. Knowledge of the Agile Development Method and the Scrum agile framework is also necessary for the user of this product. Documentation concerning ScrumBot and its commands will be provided to the users.

\subsection{Waiting Room}
There are no requirements in the waiting room.

\subsection{Ideas for Solutions}
N/A

\bibliographystyle{plainnat}

\bibliography{SRS}

\newpage

\section{Appendix}
% This section has been added to the Volere template.  This is where you can place additional information.

\subsection{Symbolic Parameters}
% The definition of the requirements will likely call for SYMBOLIC\_CONSTANTS. Their values are defined in this section for easy maintenance.


\end{document}
