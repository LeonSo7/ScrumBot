% !TeX root = ./SRS.tex
\documentclass[12pt, titlepage]{article}

\usepackage{booktabs}
\usepackage{tabularx}
\usepackage{hyperref}
\hypersetup{
    colorlinks,
    citecolor=black,
    filecolor=black,
    linkcolor=red,
    urlcolor=blue
}
\usepackage[round]{natbib}

\title{SE 3XA3: Software Requirements Specification\\ScrumBot}

\author{
    Team 304, ScrumBot
        \\ Arkin Modi, modia1
        \\ Leon So, sol4
        \\ Timothy Choy, choyt2
}
\date{Last Updated: February 3, 2020}

%\input{../Comments}

\begin{document}

\maketitle

\pagenumbering{roman}
\tableofcontents
\listoftables
\listoffigures

\begin{table}[!hbp]
    \caption{Revision History} \label{TblRevisionHistory}
    \begin{tabularx}{\textwidth}{llX}
        \toprule
            \textbf{Date} & \textbf{Developer(s)} & \textbf{Change}\\
        \midrule
            January 23, 2020 & Arkin Modi & Copy template\\
            February 3, 2020 & Leon So & Worked on Stakeholders section\\
            February 4, 2020 & Leon So & Worked on Purpose, Non-functional Requirements sections\\
            February 8, 2020 & Leon So & Formatted and added non-functional requirements; Created terminology table; worked on Project Scope; Relevant Facts and Assumptions\\
            February 8, 2020 & Arkin Modi & Worked on Mandated Constraints, Off-the-Shelf Solutions\\
            February 8, 2020 & Timothy Choy & Worked on Look and Feel Requirements, Legal Requirements\\
            February 8, 2020 & Timothy Choy & Updated SRS formatting\\
            February 8, 2020 & Leon So & Add Functional Requirements, Update Stakeholders\\
        \bottomrule
    \end{tabularx}
\end{table}

\newpage

\pagenumbering{arabic}

This document describes the requirements for Scrumbot.  The template for the Software
Requirements Specification (SRS) is a subset of the Volere.
template~\citep{RobertsonAndRobertson2012}.  If you make further modifications
to the template, you should explicitly state what modifications were made.

\section{Project Drivers}

\subsection{The Purpose of the Project}
Scrum is an agile process framework widely used in industry for managing and coordinating collaborative projects. Scrum being a process based on the agile development method, follows a highly iterative process and often has heavy customer involvement, therefore it can be often be complex. It is often difficult to manage communication and coordination within the project due to the complexity of agile development processes. With Discord being a popular communication tool used by many teams of software developers today, Scrumbot provides a solution that directly integrates the management of a scrum development cycle into the communication channels. Scrumbot will allow for better management and organization of retrospectives, stand-ups, and other scrum/agile processes used by software teams during routine communication.\\

As a result, Scrumbot, a well-integrated scrum management system, can reduce inefficiencies surrounding the management of projects using the scrum framework, as well as strengthen communication within the agile process. This will help software development firms reduce costs and achieve better workflow.

\subsection{The Stakeholders}
This solution will be used in an environment where the development team uses the scrum agile framework and uses Discord as their routine communication channel.
\subsubsection{The Client}
\begin{itemize}
    \item Dr. Ashgar Bokhari (3XA3)
    \begin{itemize}
      \item[] The client for this project is Dr. Ashgar Bokhari - professor of the 3XA3 course.
    \end{itemize}
\end{itemize}
\subsubsection{The Customer}
The customer of the ScrumBot is any firm looking to improve efficiency in communication and organization during the Scrum process of an agile software development cycle using the Scrum framework. This customer will likely be a software development firm (or the development department of a firm).
\begin{itemize}
    \item Software Development Firm
    \begin{itemize}
      \item[] The development firm will benefit through centralizing project resources, therefore improving the organization and management of the project. The development firm will therefore be able to reduce any costs associated with inefficiencies during the development process. This will also improve communication between members of the development firm.
    \end{itemize}
\end{itemize}
\subsubsection{The Users (Roles in Scrum process)}
\begin{itemize}
    \item Product Owner(s)
    \begin{itemize}
      \item[] The product owner(s) will benefit from the reduced development time through improved communication and efficient management of the project. The product owner(s) will also benefit as their ideas and requirements will be better organized, thus helping the development team more efficiently achieve the goals of the product owner(s).
    \end{itemize}
    \item Scrum Master
    \begin{itemize}
      \item[] The scrum master will be better able to coordinate scrum plans and division of tasks, as well as routine communication (i.e. retrospectives and stand-ups). With a scrum management system directly implemented into the communication channels, the scrum master will be able to directly manage the meeting information and their scrum development cycle during communication.
    \end{itemize}
    \item Development Team
    \begin{itemize}
      \item[] The development team will be able to more easily follow through on tasks assigned by the scrum master. The development team may easily refer to any relevant articles or information regarding their project. This will reduce inefficiencies and speedup communication within the development team. It will also reduce confusion, as all information concerning the development cycle will be available through their routine communication channel (Discord).
    \end{itemize}
    \item Business Analyst
    \begin{itemize}
      \item[] The Business Analyst will be able to better coordinate communication concerning the project between the development team and the product owner(s). ScrumBot will also allow the Business Analyst to better: organize and monitor the progress of the project, update requirements, and analyse sprint progress.
    \end{itemize}
\end{itemize}
\subsubsection{Other stakeholders}


\subsection{Mandated Constraints}
\subsubsection{Solution Constraints}
\noindent \textit{Description:} The project shall be written in Python 3.\\
\textit{Rationale:} All developers are familiar with Python 3 and there is an existing API available in Python 3.\\
\textit{Fit Criterion:} The software is written in Python 3.\\

\noindent \textit{Description:} The application shall function within Discord.\\
\textit{Rationale:} Discord is a popular communication tool used by many software development team. In addition, the existing project utilizes Discord to perform all of its features.\\
\textit{Fit Criterion:} All application features are fully operational within Discord.

\subsubsection{Implementation Environment of the Current System}
The application will be be installed on the user's local machine or a server, and added to a Discord server. To function, the application be started on the machine and the user will interact with it through the respective Discord server.

\subsubsection{Partner or Collaborative Applications}
The application will use Discord and the Discord API to function as an interface.

\subsubsection{Off-the-Shelf Software}
The following off-the-shelf software will utilized:
\begin{itemize}
    \item Discord (Available at \url{https://discordapp.com/})
    \item Discord Server (Available through Discord)
    \item Python 3 (Available at \url{https://www.python.org/downloads/})
    \item discord.py (Available at \url{https://pypi.org/project/discord.py/})
\end{itemize}
All off-the-shelf software are available for free.

\subsubsection{Anticipated Workplace Environment}
The anticipated workplace environment for this application is within a software development team's Discord server. This application can be used from anywhere, as long as the user has access to the respective Discord server.

\subsubsection{Schedule Constraints}
The project deliverable must completed by their respective deadline. The remaining deadline include:
\begin{itemize}
    \item Proof of Concept (February 13, 2020, 1:00 PM)
    \item Test Plan, Revision 0 (February 28, 2020, 11:30 PM)
    \item Design \& Document, Revision 0 (March 13, 2020, 11:30 PM)
    \item Demonstration, Revision 0 (March 17, 2020)
    \item Demonstration, Revision 1 (March 31, 2020)
    \item Final Documentation, Revision 1 (April 6, 2020, 11:30 PM)
\end{itemize}

\subsubsection{Budget Constraints}
This project has no monetary budget. If there are any necessary purchases for development, the cost shall be paid by the project members. All resources to re-create and upgrade the existing project are provided.

\subsubsection{Enterprise Constraints}
This application will be available for free to any user that has access to a Discord server and a machine with Python 3 (and the dependencies) installed.

\subsection{Naming Conventions and Terminology}

\subsection{Relevant Facts and Assumptions}

It is assumed that the user will be deploying the Discord bot on Discord. It is assumed that the user is familiar with the Scrum Agile process framework and the Agile development software design methodology.

\section{Functional Requirements}

\subsection{The Scope of the Work and the Product}

\subsubsection{The Context of the Work}
This product is designed to be used by software development teams using: Discord as their main communication channel, an Agile development software design methodology, and Scrum agile process framework. Additionally, this application can be intregated with Trello, and Google Calendar and Maps.

\subsubsection{Work Partitioning}

\subsubsection{Individual Product Use Cases}

\subsection{Functional Requirements}
\begin{enumerate}[{BE}1.]
    \item The development firm wants to add ScrumBot to its Discord channel
    \begin{enumerate}[{VP}1.]
        \item Viewpoint:
    \end{enumerate}

    \item The Business Analyst wants to add a new project
    \begin{enumerate}[{VP}1.] 
        \item Viewpoint: Development Team
            \begin{enumerate}
                \item The system shall update the project list of the development team
                \item The system shall notify the development team of the new project
                \item The system shall notify the development team of the scheduled first meeting
            \end{enumerate}
        \item Viewpoint: Business Analyst
            \begin{enumerate}
                \item The system shall prompt the Business Analyst for the project name
                \item The system shall prompt the Business Analyst for the project details
                \item The system shall prompt the Business Analyst for first meeting details
                \item The system shall append the project and its details to the project list
            \end{enumerate}
        \item Viewpoint: Scrum Master
            \begin{enumerate}
                \item The system shall update the project list of the Scrum Master
                \item The system shall notify the Scrum Master of the new project
                \item The system shall notify the Scrum Master of the scheduled first meeting
            \end{enumerate}
    \end{enumerate}
\end{enumerate}

\section{Non-functional Requirements}

\subsection{Look and Feel Requirements}
\begin{itemize}
    \item The system will output the 
\end{itemize}

\subsubsection{Style Requirements}
\begin{itemize}
    \item The system shall use short, clear, and descriptive role names
\end{itemize}

\subsection{Usability and Humanity Requirements}
\subsubsection{Ease of Use Requirements}
\begin{itemize}
    \item The system shall have intuitive user commands
    \item The system shall be easy to use for ages 13+
\end{itemize}
\subsubsection{Personalization and Internationalization Requirements}
\begin{itemize}
    \item The system shall be used by English speakers only
\end{itemize}
\subsubsection{Learning Requirements}
\begin{itemize}
    \item The system shall have a help menu to explain commands to the user
\end{itemize}

\subsection{Performance Requirements}
\subsubsection{Speed and Latency Requirements}
\begin{itemize}
    \item The system shall respond to user commands within 2 seconds.
\end{itemize}
\subsubsection{Precision or Accuracy Requirements}
\begin{itemize}
    \item The system shall have a meeting locations and schedules accuracy greater than 70\%
    \item All meeting times should be accurate to within 30 seconds
\end{itemize}
\subsubsection{Reliability and Availability Requirements}
\begin{itemize}
    \item The system shall load and use the correct data for the team
\end{itemize}

\subsection{Operational and Environmental Requirements}
\subsubsection{Expected Environment}
\begin{itemize}
    \item The system shall be able to operated and be executed within the Discord application
\end{itemize}
\subsubsection{Requirements for Interfacing with Adjacent Systems}
\begin{itemize}
    \item The system shall connect to Google's API services
    \item The system shall connect to Discord's API services
    \item The system shall connect to Trello's API services
\end{itemize}

\subsection{Maintainability and Support Requirements}
\subsubsection{Maintainability Requirements}
\begin{itemize}
    \item The code shall be documented using comments
    \item The code shall be documented using Doxygen
    \item The code documentation shall be easy to understand
\end{itemize}
\subsubsection{Supportability Requirements}
\begin{itemize}
    \item The system shall have a help menu available to users at all times
\end{itemize}
\subsubsection{Longevity Requirements}
\begin{itemize}
    \item The system will be made in modules to increase maintainability and longevity
\end{itemize}

\subsection{Security Requirements}
\begin{itemize}
    \item The connection between the system and the APIs shall use HTTPS for security
\end{itemize}

\subsection{Cultural Requirements}
\begin{itemize}
    \item The system shall use Canadian English spelling
\end{itemize}

\subsection{Legal Requirements}
N/A

\subsection{Health and Safety Requirements}
N/A

\section{Project Issues}

\subsection{Open Issues}

\subsection{Off-the-Shelf Solutions}

\subsection{New Problems}

\subsection{Tasks}

\subsection{Migration to the New Product}

\subsection{Risks}

\subsection{Costs}

\subsection{User Documentation and Training}

\subsection{Waiting Room}

\subsection{Ideas for Solutions}

\bibliographystyle{plainnat}

\bibliography{SRS}

\newpage

\section{Appendix}

This section has been added to the Volere template.  This is where you can place
additional information.

\subsection{Symbolic Parameters}

The definition of the requirements will likely call for SYMBOLIC\_CONSTANTS.
Their values are defined in this section for easy maintenance.


\end{document}
