% !TeX root = ./SRS.tex
\documentclass[12pt, titlepage]{article}

\usepackage{booktabs}
\usepackage{tabularx}
\usepackage{hyperref}
\hypersetup{
    colorlinks,
    citecolor=black,
    filecolor=black,
    linkcolor=red,
    urlcolor=blue
}
\usepackage[round]{natbib}

\title{SE 3XA3: Software Requirements Specification\\ScrumBot}

\author{
    Team 304, ScrumBot
        \\ Arkin Modi, modia1
        \\ Leon So, sol4
        \\ Timothy Choy, choyt2
}
\date{Last Updated: February 3, 2020}

%\input{../Comments}

\begin{document}

\maketitle

\pagenumbering{roman}
\tableofcontents
\listoftables
\listoffigures

\begin{table}[bp]
    \caption{Revision History} \label{TblRevisionHistory}
    \begin{tabularx}{\textwidth}{llX}
        \toprule
            \textbf{Date} & \textbf{Developer(s)} & \textbf{Change}\\
        \midrule
            January 23, 2020 & Arkin Modi & Copy template\\
            February 3, 2020 & Leon So & Worked on Stakeholders section\\
            February 4, 2020 & Leon So & Worked on Purpose, Non-functional Requirements sections\\
            February 8, 2020 & Leon So & Formatted and added non-functional requirements\\
        \bottomrule
    \end{tabularx}
\end{table}

\newpage

\pagenumbering{arabic}

This document describes the requirements for Scrumbot.  The template for the Software
Requirements Specification (SRS) is a subset of the Volere.
template~\citep{RobertsonAndRobertson2012}.  If you make further modifications
to the template, you should explicitly state what modifications were made.

\section{Project Drivers}

\subsection{The Purpose of the Project}
Scrum is an agile process framework widely used in industry for managing and coordinating collaborative projects. Scrum being a process based on the agile development method, follows a highly iterative process and often has heavy customer involvement, therefore it can be often be complex. It is often difficult to manage communication and coordination within the project due to the complexity of agile development processes. With Discord being a popular communication tool used by many teams of software developers today, Scrumbot provides a solution that directly integrates the management of a scrum development cycle into the communication channels. Scrumbot will allow for better management and organization of retrospectives, stand-ups, and other scrum/agile processes used by software teams during routine communication.\\

As a result, Scrumbot, a well-integrated scrum management system, can reduce inefficiencies surrounding the management of projects using the scrum framework, as well as strengthen communication within the agile process. This will help software development firms reduce costs and achieve better workflow.

\subsection{The Stakeholders}
This solution will be used in an environment where the development team uses the scrum agile framework and uses Discord as their routine communication channel.
\subsubsection{The Client}
\begin{itemize}
    \item Dr. Ashgar Bokhari (3XA3)
    \begin{itemize}
      \item[] The client for this project is Dr. Ashgar Bokhari - professor of the 3XA3 course.
    \end{itemize}
\end{itemize}
\subsubsection{The Customer}
The customer of the ScrumBot is any firm looking to improve efficiency in communication and organization during the Scrum process of an agile software development cycle using the Scrum framework. This customer will likely be a software development firm (or the development department of a firm).
\begin{itemize}
    \item Software Development Firm
    \begin{itemize}
      \item[] The development firm will benefit through centralizing project resources, therefore improving the organization and management of the project. The development firm will therefore be able to reduce any costs associated with inefficiencies during the development process. This will also improve communication between members of the development firm.
    \end{itemize}
\end{itemize}
\subsubsection{The Users (Roles in Scrum process)}
\begin{itemize}
    \item Product Owner(s)
    \begin{itemize}
      \item[] The product owner(s) will benefit from the reduced development time through improved communication and efficient management of the project. The product owner(s) will also benefit as their ideas and requirements will be better organized, thus helping the development team more efficiently achieve the goals of the product owner(s).
    \end{itemize}
    \item Scrum Master
    \begin{itemize}
      \item[] The scrum master will be better able to coordinate scrum plans and division of tasks, as well as routine communication (i.e. retrospectives and stand-ups). With a scrum management system directly implemented into the communication channels, the scrum master will be able to directly manage the meeting information and their scrum development cycle during communication.
    \end{itemize}
    \item Development Team
    \begin{itemize}
      \item[] The development team will be able to more easily follow through on tasks assigned by the scrum master. The development team may easily refer to any relevant articles or information regarding their project. This will reduce inefficiencies and speedup communication within the development team. It will also reduce confusion, as all information concerning the development cycle will be available through their routine communication channel (Discord).
    \end{itemize}
\end{itemize}
\subsubsection{Other stakeholders}


\subsection{Mandated Constraints}
\subsubsection{Solution Constraints}

\subsubsection{Implementation Environment of the Current System}
The application will be be installed on the user's local machine or a server, and added to a Discord server. To function, the application be started on the machine and the user will interact with it through the respective Discord server.

\subsubsection{Partner or Collaborative Applications}

\subsubsection{Off-the-Shelf Software}

\subsubsection{Anticipated Workplace Environment}

\subsubsection{Schedule Constraints}

\subsubsection{Budget Constraints}

\subsubsection{Enterprise Constraints}

\subsection{Naming Conventions and Terminology}

\subsection{Relevant Facts and Assumptions}

User characteristics should go under assumptions.

\section{Functional Requirements}

\subsection{The Scope of the Work and the Product}

\subsubsection{The Context of the Work}

\subsubsection{Work Partitioning}

\subsubsection{Individual Product Use Cases}

\subsection{Functional Requirements}

\section{Non-functional Requirements}

\subsection{Look and Feel Requirements}
\begin{itemize}
    \item The system will output the 
\end{itemize}

\subsection{Usability and Humanity Requirements}
\begin{itemize}
    \item The system shall have intuitive user commands
\end{itemize}

\subsection{Performance Requirements}
\subsubsection{Speed and Latency Requirements}
\begin{itemize}
    \item The system shall respond to user commands within 2 seconds.
\end{itemize}
\subsubsection{Precision or Accuracy Requirements}
\begin{itemize}
    \item The system shall have a meeting locations and schedules accuracy greater than 70\%.
\end{itemize}

\subsection{Operational and Environmental Requirements}
\subsubsection{Expected Environment}
\begin{itemize}
    \item The system shall be able to operated and be executed within the Discord application
\end{itemize}
\subsubsection{Requirements for Interfacing with Adjacent Systems}
\begin{itemize}
    \item The system shall connect to Google API services
\end{itemize}

\subsection{Maintainability and Support Requirements}
\begin{itemize}
    \item The code shall be documented using comments
    \item The code shall be documented using Doxygen
    \item The code documentation shall be easy to understand
\end{itemize}

\subsection{Security Requirements}
\begin{itemize}
    \item The connection between the system and the APIs shall use HTTPS for security
\end{itemize}

\subsection{Cultural Requirements}
\begin{itemize}
    \item The system shall use Canadian English spelling
\end{itemize}

\subsection{Legal Requirements}

\subsection{Health and Safety Requirements}

This section is not in the original Volere template, but health and safety are
issues that should be considered for every engineering project.

\section{Project Issues}

\subsection{Open Issues}

\subsection{Off-the-Shelf Solutions}

\subsection{New Problems}

\subsection{Tasks}

\subsection{Migration to the New Product}

\subsection{Risks}

\subsection{Costs}

\subsection{User Documentation and Training}

\subsection{Waiting Room}

\subsection{Ideas for Solutions}

\bibliographystyle{plainnat}

\bibliography{SRS}

\newpage

\section{Appendix}

This section has been added to the Volere template.  This is where you can place
additional information.

\subsection{Symbolic Parameters}

The definition of the requirements will likely call for SYMBOLIC\_CONSTANTS.
Their values are defined in this section for easy maintenance.


\end{document}
