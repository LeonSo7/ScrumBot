\documentclass[12pt, titlepage]{article}

\usepackage{booktabs}
\usepackage{tabularx}
\usepackage{hyperref}
\hypersetup{
    colorlinks,
    citecolor=black,
    filecolor=black,
    linkcolor=red,
    urlcolor=blue
}
\usepackage[round]{natbib}
\usepackage{float}
\usepackage{setspace}
\usepackage{xspace}

% Document margins
\usepackage[left=1in,top=1in,right=1in,bottom=1in]{geometry}

\title{SE 3XA3: Test Plan\\ScrumBot}

\author{
	Team 304, ScrumBot
		\\ Arkin Modi, modia1
        \\ Leon So, sol4
        \\ Timothy Choy, choyt2
}
\date{Last Updated: February 28, 2020}

%\input{../Comments}

\begin{document}
\newpage
\maketitle

\pagenumbering{roman}
\tableofcontents
\listoftables
\listoffigures

\newpage

\begin{table}[!hbp]
    \caption{Revision History} \label{TblRevisionHistory}
    \begin{tabularx}{\textwidth}{llX}
        \toprule
            \textbf{Date} & \textbf{Developer(s)} & \textbf{Change}\\
        \midrule
            January 23, 2020 & Arkin Modi & Copy template\\
            February 20, 2020 & Arkin Modi & Created the Purpose Section\\
            February 27, 2020 & Timothy Choy & Worked on Scope and Acronyms and Abbreviations\\
            February 27, 2020 & Leon So & Worked on Scope and Overview of Document\\
            February 27, 2020 & Leon So & Worked on Software Description\\
            February 27, 2020 & Arkin Modi & Worked on Test Team, Automated Testing Approach, and Testing Tools\\
            February 27, 2020 & Leon So & Worked on Tests for NFRs\\
            February 27, 2020 & Arkin Modi & Worked on the Unit Testing Plan for internal functions and output files\\
            February 28, 2020 & Timothy Choy & Worked on the Proof of Concept Testing\\
            February 28, 2020 & Leon So & Worked on Tests for NFRs\\
            February 28, 2020 & Timothy Choy & Updated Definitions, Worked on Tests for FRs\\
        \bottomrule
    \end{tabularx}
\end{table}


\newpage

\pagenumbering{arabic}

\section{General Information}

\subsection{Purpose}
The purpose of this document is to outline the testing, validation, and verification process of the functional and non-functional requirements, for the ScrumBot project. These test cases were conceived before the implementation and therefore will be used by the project members for future reference during the development process.

\subsection{Scope}
% The test plan outlines all tests which will be conducted during this project, and the testing tools that will be used.
% The scope of the test plan, what and how we are going to do this
This test plan will provide a method to fully test ScrumBot by performing tests both at a modular level using unit tests created through Pytest, as well as at higher level, through using exploratory testing and specification-based testing. The unit tests cases will also cover partition testing, fuzz testing, and boundary testing.

\subsection{Acronyms, Abbreviations, and Symbols}
	
\begin{table}[hbp]
    \caption{Table of Abbreviations}
    \label{Table}
    \begin{tabularx}{\textwidth}{p{3cm}X}
        \toprule
        \textbf{Abbreviation} & \textbf{Definition} \\
        \midrule
        CD & Continuous Delivery/Deployment\\
        CI & Continuous Integration\\
        EDT & Eastern Daylight Time (UTC-4)\\
        EST & Eastern Standard Time (UTC-5)\\
        FR & Functional Requirement\\
        HTTP & HyperText Transfer Protocol\\
        MVC & Model View Controller\\
        NFR & Non-functional Requirement\\
        POC & Proof of Concept\\
        SRS & Software Requirements Specification\\
        UTC & Coordinated Universal Time\\
        \bottomrule
    \end{tabularx}
\end{table}
\newpage
\begin{table}[!htbp]
    \caption{Table of Definitions}
    \label{Table}
    \begin{tabularx}{\textwidth}{p{4cm}X}
        \toprule
        \textbf{Term} & \textbf{Definition}\\
        \midrule
        Acceptance Testing & A method of testing which is conducted to determine if the requirements of the specification are met\\
        Boundary Testing & A method of testing where values are chosen on semantically significant boundaries\\
        Code Inspection & A method of static testing where developers walk through the code\\
        Discord & A chat application. The platform in which ScrumBot will be implemented.\\
        Dynamic Testing & A method of testing where code is executed\\
        Exploratory Testing & A method of testing where the tester simultaneously learns the code while testing it. It approaches testing from a user's viewpoint\\
        Fuzz Testing & A method of testing where random inputs are given to attempt to violate assertions\\
        Integration Testing & A method of testing where individual software modules are combined and tested as a group\\
        Kanban Board & A method of scheduling tasks through categorizing tasks to improve efficiency\\
        Partition Testing & A method of testing where the input domain is partitioned and input values are selected from the partitions\\
        Pylint & A Python linter, used for static testing\\
        Pytest & A unit testing framework for Python\\
        ScrumBot & The Discord bot in development\\
        Specification-based Testing & A method of testing where test cases are built based on the requirements specification\\
        Static Testing & A method of testing where code is not executed\\
        System Testing & A method of testing where the tests are performed on the system as a whole\\
        Trello & A web based Kanban project management system\\
        Unit Testing & A method of testing focused on testing individual methods and functions\\
        \bottomrule
    \end{tabularx}
\end{table}	

\subsection{Overview of Document}
% The overview of this document
This document outlines a test plan that fully encompasses all requirements of ScrumBot, as stated in the SRS. This document includes relevant information concerning: test team, automated testing, testing tools, testing schedule, unit-testing, and test cases.

\section{Plan}
\subsection{Software Description}
Scrum is an Agile process framework widely used in industry for managing and coordinating collaborative projects. Scrum being a process based on the agile development method, follows a highly iterative process and often has heavy customer involvement, therefore it can be often be complex. With Discord being a popular communication tool used by many teams of software developers today, ScrumBot provides a solution that directly integrates the management of a scrum development cycle into the communication channels. ScrumBot will allow for better management and organization of retrospectives, stand-ups, and other scrum/agile stages used by software teams within their routine communication channel.

\subsection{Test Team}
The test team will consist of all the members of the project: Arkin Modi, Leon So, and Timothy Choy.

\subsection{Automated Testing Approach}
Testing shall be automated with the use of the GitLab's CI/CD tool and the Pytest framework. The tests will be run every time a commit is pushed to the repository.

\subsection{Testing Tools}
The unit tests will be written using the Pytest framework. Static testing will be done with the use of Pylint.

\subsection{Testing Schedule}
See Gantt Chart at the following URL, \url{https://gitlab.cas.mcmaster.ca/modia1/ScrumBot/-/blob/master/ProjectSchedule/}.

\section{System Test Description}
\subsection{Tests for Functional Requirements}

\subsubsection{Installation}

\begin{enumerate}
    \item{\textbf{BE1-1}}\\
    Type: Specification-based, Dynamic, Manual\\
    Initial State: A Discord channel is active\\
    Input: ScrumBot is added into the channel\\
    Output: Various notifications, such as a link to documentation and basic commands\\
    How test will be performed: ScrumBot will be manually added to a Discord channel\\
\end{enumerate}

\subsubsection{Project Creation}

\begin{enumerate}
    \item{\textbf{BE2-1}}\\
    Type: Specification-based, Dynamic, Manual, Static etc.\\
    Initial State: \\
    Input: \\
    Output: \\
    How test will be performed: \\
\end{enumerate}

\subsubsection{Project Removal}

\begin{enumerate}
    \item{\textbf{BE3-1}}\\
    Type: Specification-based, Dynamic, Manual, Static etc.\\
    Initial State: \\
    Input: \\
    Output: \\
    How test will be performed: \\
\end{enumerate}

\subsubsection{Sprint-planning Meeting}

\begin{enumerate}
    \item{\textbf{BE4-1}}\\
    Type: Specification-based, Dynamic, Manual, Static etc.\\
    Initial State: \\
    Input: \\
    Output: \\
    How test will be performed: \\
\end{enumerate}

\subsubsection{Stand-up Meeting}

\begin{enumerate}
    \item{\textbf{BE5-1}}\\
    Type: Specification-based, Dynamic, Manual, Static etc.\\
    Initial State: \\
    Input: \\
    Output: \\
    How test will be performed: \\
\end{enumerate}

\subsubsection{Retrospective Meeting}

\begin{enumerate}
    \item{\textbf{BE6-1}}\\
    Type: Specification-based, Dynamic, Manual, Static etc.\\
    Initial State: \\
    Input: \\
    Output: \\
    How test will be performed: \\
\end{enumerate}

\subsubsection{Grooming Meeting}

\begin{enumerate}
    \item{\textbf{BE7-1}}\\
    Type: Specification-based, Dynamic, Manual, Static etc.\\
    Initial State: \\
    Input: \\
    Output: \\
    How test will be performed: \\
\end{enumerate}

\subsubsection{Add a Meeting}

\begin{enumerate}
    \item{\textbf{BE8-1}}\\
    Type: Dynamic, Unit\\
    Initial State: ScrumBot active in Discord channel\\
    Input: Pytest file is run\\
    Output: Pytest output\\
    How test will be performed: The test will be performed through Pytest, where tests will run the command to add meetings, focusing on the parameters entered to create meetings\\
    
    \item{\textbf{BE8-2}}\\
    Type: Specification-based, Dynamic, Boundary, Manual\\
    Initial State: ScrumBot active in Discord channel\\
    Input: Cancel meeting command is entered, with a specified meeting\\
    Output: The specified meeting will be removed, otherwise an error message is raised\\
    How test will be performed: The test will be performed manually through commands entered in Discord. Manual testing primarily will focus on if notifications are sent to the proper people (development team members, business analysts and scrum masters), as well as the interactions between ScrumBot and the tester\\
\end{enumerate}

\subsubsection{Cancel a Meeting}

\begin{enumerate}
    \item{\textbf{BE9-1}}\\
    Type: Dynamic, Boundary, Unit\\
    Initial State: ScrumBot active in Discord channel\\
    Input: Pytest file is run\\
    Output: Pytest output\\
    How test will be performed: The test will be performed through Pytest, where tests will run the command to cancel meetings, in situations where the meeting exists and where the meeting does not exist\\
    
    \item{\textbf{BE9-2}}\\
    Type: Specification-based, Dynamic, Manual\\
    Initial State: ScrumBot active in Discord channel\\
    Input: Cancel meeting command is entered, with a specified meeting\\
    Output: The specified meeting will be removed, otherwise an error message is raised\\
    How test will be performed: The test will be performed manually through commands entered in Discord. Manual testing primarily will focus on if notifications are sent to the proper people (development team members, business analysts and scrum masters)\\
\end{enumerate}

\subsubsection{List Scheduled Meetings}

\begin{enumerate}
    \item{\textbf{BE10-1}}\\
    Type: Specification-based, Dynamic, Boundary, Unit\\
    Initial State: ScrumBot active in Discord channel\\
    Input: Pytest file is run\\
    Output: Pytest output\\
    How test will be performed: The test will be performed through Pytest, where tests will run the command to list meetings, in situations where there are meetings scheduled and when there are no meetings\\
    
    \item{\textbf{BE10-2}}\\
    Type: Specification-based, Dynamic, Boundary, Manual\\
    Initial State: ScrumBot active in Discord channel\\
    Input: The list scheduled meetings command is entered into Discord\\
    Output: ScrumBot should list out the list of scheduled meetings\\
    How test will be performed: The test will be performed manually through commands entered in Discord. Test cases be similar to the unit tests above\\
\end{enumerate}

\subsubsection{See Tasks}

\begin{enumerate}
    \item{\textbf{BE11-1}}\\
    Type: Specification-based, Dynamic, Manual\\
    Initial State: ScrumBot active in Discord channel\\
    Input: The see tasks command is entered into Discord\\
    Output: A link to the Trello Kanban board, if integrated\\
    How test will be performed: The test will be performed manually through commands entered in Discord.\\
\end{enumerate}

\subsection{Tests for Non-Functional Requirements}

\subsubsection{Usability and Humanity Requirements}
\paragraph{Learning Requirements}
\begin{enumerate}
\item{\textbf{NFRT-UH4}}\\
Type: Dynamic, Manual\\
Initial State: ScrumBot active in Discord channel\\
Input: Help command entered\\
Output: ScrumBot should output help menu\\
How test will be performed: The user will enter help menu into Discord chat with ScrumBot active
\end{enumerate}

\subsubsection{Performance Requirements}
\paragraph{Response Speed}
\begin{enumerate}
\item{\textbf{NFRT-P1}}\\
Type: Dynamic, Manual\\
Initial State: No commands being made\\
Input: Command entered\\
Output: Response should be received within 2 seconds of the input being sent\\
How test will be performed: The user will enter a command into the Discord channel with ScrumBot active. ScrumBot should provide a response within 2ms of the command being entered
\end{enumerate}
\paragraph{Meeting Schedule Accuracy}
\begin{enumerate}
\item{\textbf{NFRT-P2}}\\
Type: Dynamic, Manual\\
Initial State: No meetings schedules\\
Input: Schedule meeting command with meeting details\\
Output: The meeting is added to the meeting list with the correct location\\
How test will be performed: The user will enter the command to schedule a meeting and the meeting details into the Discord channel with ScrumBot active
\item{\textbf{NFRT-P3}}\\
Type: Dynamic, Manual\\
Initial State: No meetings schedules\\
Input: Schedule meeting command with meeting details\\
Output: The meeting is added to the meeting list with the correct time\\
How test will be performed: The user will enter the command to schedule a meeting and the meeting details into the Discord channel with ScrumBot active
\end{enumerate}

\subsubsection{Operational and Environmental Requirements}
\paragraph{Expected Environment}
\begin{enumerate}

\item{\textbf{NFRT-OE1}}\\
Type: Dynamic, Manual\\
Initial State: New Discord channel without ScrumBot\\
Input/Condition: Add ScrumBot to the Discord channel\\
Output/Result: ScrumBot should be fully functional in the Discord channel once added\\
How test will be performed: The test team will follow the provided installation documentation and add ScrumBot to a brand new Discord channel
\end{enumerate}
\paragraph{Requirements for Interfacing with Adjacent Systems}
\begin{enumerate}

\item{\textbf{NFRT-OE2}}\\
Type: Dynamic, Manual\\
Initial State: ScrumBot added to Discord channel not yet connected to users' Google services\\
Input/Condition: User wants to connect their Google services to ScrumBot\\
Output/Result: ScrumBot should be connect to the User's Google account through Google API services\\
How test will be performed: The test team will attempt to connect their Google services to ScrumBot

\item{\textbf{NFRT-OE3}}\\
Type: Dynamic, Manual\\
Initial State: ScrumBot not yet registered as a public bot\\
Input/Condition: ScrumBot register with Discord API services and deployed publically\\
Output/Result: ScrumBot should be a public bot registered with Discord\\
How test will be performed: The test team will follow documentation provided by Discord to add ScrumBot as a public bot

\item{\textbf{NFRT-OE4}}\\
Type: Dynamic, Manual\\
Initial State: ScrumBot added to Discord channel not yet connected to users' Trello services\\
Input/Condition: User wants to connect their Trello services to ScrumBot\\
Output/Result: ScrumBot should be connect to the User's Trello account through Trello API services\\
How test will be performed: The test team will attempt to connect their Trello services to ScrumBot
\end{enumerate}

\paragraph{Installability Requirements}
\begin{enumerate}

\item{\textbf{NFRT-OE5}}\\
Type: Dynamic, Manual\\
Initial State: ScrumBot not yet added to Discord channel\\
Input/Condition: User without expertise in Discord bots wants to install ScrumBot\\
Output/Result: ScrumBot added to Discord channel\\
How test will be performed: An inexperienced user will attempt to add ScrumBot to a Discord channel by following the provided documentation

\item{\textbf{NFRT-OE6}}\\
Type: Dynamic, Manual\\
Initial State: ScrumBot not yet added to Discord channel\\
Input/Condition: Start installation of ScrumBot\\
Output/Result: ScrumBot takes less than 5 minutes to install\\
How test will be performed: Test team will install ScrumBot and time the installation

\item{\textbf{NFRT-OE7}}\\
Type: Dynamic, Manual\\
Initial State: ScrumBot installed\\
Input/Condition: Start uninstalling Scrumbot\\
Output/Result: ScrumBot is uninstalled\\
How test will be performed: The test team will attempt to uninstall ScrumBot
\end{enumerate}


\subsubsection{Maintainability and Support Requirements}
\paragraph{Maintainability Requirements}
\begin{enumerate}

\item{\textbf{NFRT-MS1}}\\
Type: Static, Manual, Code Inspection\\
Initial State: N/A\\
Input/Condition: N/A\\
Output/Result: The code is well documented with comments\\
How test will be performed: The test team will inspect the code and check if the code is adequately documented using comments
\item{\textbf{NFRT-MS2}}\\
Type: Static, Manual\\
Initial State: No Doxygen documents generated yet\\
Input/Condition: Generate Doxygen HTML and pdf\\
Output/Result: The Doxygen documentation is successfully compiled and generated\\
How test will be performed: The test team will try to generate Doxygen HTML and pdf
\end{enumerate}

\paragraph{Supportability Requirements}
\begin{enumerate}

\item{\textbf{NFRT-MS4}}\\
Type: Dynamic, Manual\\
Initial State: ScrumBot is active on the Discord channel\\
Input/Condition: User wants to open help menu\\
Output/Result: Help menu is displayed in the Discord chat\\
How test will be performed: The test team will attempt to open the help menu in the Discord chat
\end{enumerate}

\subsubsection{Security Requirements}
\paragraph{HTTP Connections}
\begin{enumerate}

\item{\textbf{NFRT-S1}}\\
Type: Static, Manual, Code Inspection\\
Initial State: N/A\\
Input/Condition: N/A\\
Output/Result: All connections between the system and the APIs use HTTPS requests\\
How test will be performed: The test team will inspect the code and check if all connections between the system and APIs use HTTP requests
\end{enumerate}

\subsection{Traceability Between Test Cases and Requirements}
\begin{table}[H]
    \centering
    \caption{Traceability Matrix: Functional Requirement}
    \begin{tabular}{lcc}
        ~ & \textbf{FRT-??} & \textbf{FRT-??}\\
        \textbf{REQ?} & X & ~\\
        \textbf{REQ?} & ~ & X\\
    \end{tabular}
    \label{Traceability Matrix: Functional Requirement}
\end{table}

\begin{table}[H]
    \centering
    \caption{Traceability Matrix: Non-Functional Requirement}
    \begin{tabular}{lcc}
        ~ & \textbf{NFRT-P1} & \textbf{NFRT-S1}\\
        \textbf{P1} & X & ~\\
        \textbf{S1} & ~ & X\\
    \end{tabular}
    \label{Traceability Matrix: Non-Functional Requirement}
\end{table}

\section{Tests for Proof of Concept}

\subsection{Issues and Conflicts}
The proof of concept demonstration for ScrumBot consisted on a simple Python discord bot performing basic front end tasks such as creating, listing, and deleting meeting times. There was no database connected to the proof of concept, so memory was not stored from instance to instance.\\
\noindent Issues that were found with the proof of concept were:
\begin{enumerate}
    \item The lack of a priority list for features
    \item The lack of a defined software architectural style
    \item The need for time zones, as people could be connecting to meetings from around the world
\end{enumerate}

\subsection{Resolution}
To resolve these issues, we have taken the list of business events from our SRS and have assigned priority to the events. In the case where ScrumBot will not be able to fulfill all the requirements in the given timeframe, ScrumBot will still be able to function as the prioritized functionalities will be implemented.\\ \\
\noindent In regards to the chosen architectural style, we have decided on using MVC as our primary architectural style. This best suits ScrumBot as the view module will be all the input and output from Discord, our controller module will be the commands run, and our model module will be the databases containing all the meeting information.\\ \\
\noindent To tackle the issue regarding time zones, we plan on writing our times in EST or EDT, based on the date scheduled for the meeting. The choice is simply because of our current location, and makes it simpler for us to test and create meetings. However, a new feature we plan on implementing is the ability to convert between time zones given a meeting.
	
\section{Comparison to Existing Implementation}
N/A

\section{Unit Testing Plan}
Unit testing will be performed through the use of the Pytest testing framework.

\subsection{Unit testing of internal functions}
Unit testing of internal functions will be performed for every function to ensure robustness. These tests will consist of a combination of partition testing and fuzz testing. Through this, verifying that the functions react in a predictable way. Following this, the functions will then undergo integration testing, to verify the compliance of the system with the SRS. During integration testing, stubs and drivers will be created as needed. For the testing of internal functions, the team will aim for a coverage of at minimum 85\%.

\subsection{Unit testing of output files}		
The only output file will be the application executable, which shall be tested for correctness as a whole as well as the fulfillment of the SRS. This testing will take the form of system testing and acceptance testing. The system testing will test the performance, the behavior under extreme/varying load, and scalability with the number of users. The acceptance testing will ensure the SRS has been fulfilled.

\bibliographystyle{plainnat}
\bibliography{SRS}
\newpage

\section{Appendix}
% This is where you can place additional information.

% \subsection{Symbolic Parameters}
% The definition of the test cases will call for SYMBOLIC\_CONSTANTS. Their values are defined in this section for easy maintenance.

% \subsection{Usability Survey Questions?}
% This is a section that would be appropriate for some teams.

\end{document}
