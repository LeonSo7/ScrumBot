\documentclass[12pt, titlepage]{article}

\usepackage{booktabs}
\usepackage{tabularx}
\usepackage{hyperref}
\hypersetup{
    colorlinks,
    citecolor=black,
    filecolor=black,
    linkcolor=red,
    urlcolor=blue
}
\usepackage[round]{natbib}
\usepackage{float}
\usepackage{setspace}
\usepackage{xspace}

% Document margins
\usepackage[left=1in,top=1in,right=1in,bottom=1in]{geometry}

\title{SE 3XA3: Test Plan\\ScrumBot}

\author{
    Team 304, ScrumBot
        \\ Arkin Modi, modia1
        \\ Leon So, sol4
        \\ Timothy Choy, choyt2
}
\date{Last Updated: February 27, 2020}

%\input{../Comments}

\begin{document}
\newpage
\maketitle

\pagenumbering{roman}
\tableofcontents
\listoftables
\listoffigures

\newpage

\begin{table}[!hbp]
    \caption{Revision History} \label{TblRevisionHistory}
    \begin{tabularx}{\textwidth}{llX}
        \toprule
            \textbf{Date} & \textbf{Developer(s)} & \textbf{Change}\\
        \midrule
            January 23, 2020 & Arkin Modi & Copy template\\
            February 20, 2020 & Arkin Modi & Created the Purpose Section\\
            February 27, 2020 & Timothy Choy & Worked on Scope and Acronyms and Abbreviations\\
            February 27, 2020 & Leon So & Worked on Scope and Overview of Document\\
            February 27, 2020 & Leon So & Worked on Software Description\\
            February 27, 2020 & Arkin Modi & Worked on Test Team, Automated Testing Approach, and Testing Tools\\
            February 27, 2020 & Leon So & Worked on Tests for NFRs\\
            February 27, 2020 & Arkin Modi & Worked on the Unit Testing Plan for internal functions and output files\\
        \bottomrule
    \end{tabularx}
\end{table}


\newpage

\pagenumbering{arabic}

\section{General Information}

\subsection{Purpose}
The purpose of this document is to outline the testing, validation, and verification process of the functional and non-functional requirements, for the ScrumBot project. These test cases were conceived before the implementation and therefore will be used by the project members for future reference during the development process.

\subsection{Scope}
% The test plan outlines all tests which will be conducted during this project, and the testing tools that will be used.
% The scope of the test plan, what and how we are going to do this
This test plan will provide a method to fully test ScrumBot by performing tests both at a modular level using unit tests created through Pytest, as well as at higher level, through using exploratory testing and specification-based testing. The unit tests cases will also cover partition testing, fuzz testing, and boundary testing.

\subsection{Acronyms, Abbreviations, and Symbols}
    
\begin{table}[hbp]
    \caption{Table of Abbreviations}
    \label{Table}
    \begin{tabularx}{\textwidth}{p{3cm}X}
        \toprule
        \textbf{Abbreviation} & \textbf{Definition} \\
        \midrule
        CD & Continuous Delivery/Deployment\\
        CI & Continuous Integration\\
        POC & Proof of Concept\\
        SRS & Software Requirements Specification\\
        \bottomrule
    \end{tabularx}
\end{table}
\newpage
\begin{table}[!htbp]
    \caption{Table of Definitions}
    \label{Table}
    \begin{tabularx}{\textwidth}{p{4cm}X}
        \toprule
        \textbf{Term} & \textbf{Definition}\\
        \midrule
        Acceptance Testing & A method of testing which is conducted to determine if the requirements of the specification are met\\
        Boundary Testing & A method of testing where values are chosen on semantically significant boundaries\\
        Code Inspection & A method of static testing where developers walk through the code\\
        Dynamic Testing & A method of testing where code is executed\\
        Exploratory Testing & A method of testing where the tester simultaneously learns the code while testing it. It approaches testing from a user's viewpoint\\
        Fuzz Testing & A method of testing where random inputs are given to attempt to violate assertions\\
        Integration Testing & A method of testing where individual software modules are combined and tested as a group\\
        Partition Testing & A method of testing where the input domain is partitioned and input values are selected from the partitions\\
        Pylint & A Python linter, used for static testing\\
        Pytest & A unit testing framework for Python\\
        ScrumBot & The Discord bot in development\\
        Specification-based Testing & A method of testing where test cases are built based on the requirements specification\\
        Static Testing & A method of testing where code is not executed\\
        System Testing & A method of testing where the tests are performed on the system as a whole\\
        Unit Testing & A method of testing focused on testing individual methods and functions\\
        \bottomrule
    \end{tabularx}
\end{table} 

\subsection{Overview of Document}
% The overview of this document
This document outlines a test plan that fully encompasses all requirements of ScrumBot, as stated in the SRS. This document includes relevant information concerning: test team, automated testing, testing tools, testing schedule, unit-testing, and test cases.

\section{Plan}
\subsection{Software Description}
Scrum is an Agile process framework widely used in industry for managing and coordinating collaborative projects. Scrum being a process based on the agile development method, follows a highly iterative process and often has heavy customer involvement, therefore it can be often be complex. With Discord being a popular communication tool used by many teams of software developers today, ScrumBot provides a solution that directly integrates the management of a scrum development cycle into the communication channels. ScrumBot will allow for better management and organization of retrospectives, stand-ups, and other scrum/agile stages used by software teams within their routine communication channel.

\subsection{Test Team}
The test team will consist of all the members of the project: Arkin Modi, Leon So, and Timothy Choy.

\subsection{Automated Testing Approach}
Testing shall be automated with the use of the GitLab's CI/CD tool and the Pytest framework. The tests will be run every time a commit is pushed to the repository.

\subsection{Testing Tools}
The unit tests will be written using the Pytest framework. Static testing will be done with the use of Pylint.

\subsection{Testing Schedule}
See Gantt Chart at the following URL, \url{https://gitlab.cas.mcmaster.ca/modia1/ScrumBot/-/blob/master/ProjectSchedule/}.

\section{System Test Description}
\subsection{Tests for Functional Requirements}

\subsubsection{Business Event 1}
        
\paragraph{Installation}

\begin{enumerate}

\item{test-id1}\\
Type: Functional, Dynamic, Manual, Static etc.\\
Initial State: \\
Input: \\
Output: \\
How test will be performed: \\

\item{test-id2}\\
Type: Functional, Dynamic, Manual, Static etc.\\
Initial State: \\
Input: \\
Output: \\
How test will be performed: \\

\end{enumerate}

\subsubsection{Area of Testing2}

...

\subsection{Tests for Non-Functional Requirements}

\subsubsection{Performance Requirements}
\paragraph{Response Speed}
\begin{enumerate}
\item{NFRT-P1}\\
Type: Dynamic, Manual\\
Initial State: No commands being made\\
Input: Command entered\\
Output: Response should be received within 2 seconds of the input being sent
How test will be performed: The user will enter a command into the Discord channel with ScrumBot active. ScrumBot should provide a response within 2ms of the command being entered
\end{enumerate}
\paragraph{Meeting Schedule Accuracy}
\begin{enumerate}
\item{NFRT-P2}\\
Type: Dynamic, Manual\\
Initial State: No meetings schedules\\
Input: Schedule meeting command with meeting details\\
Output: The meeting is added to the meeting list with the correct location
How test will be performed: The user will enter the command to schedule a meeting and the meeting details into the Discord channel with ScrumBot active
\item{NFRT-P3}\\
Type: Dynamic, Manual\\
Initial State: No meetings schedules\\
Input: Schedule meeting command with meeting details\\
Output: The meeting is added to the meeting list with the correct time
How test will be performed: The user will enter the command to schedule a meeting and the meeting details into the Discord channel with ScrumBot active
\end{enumerate}

\subsubsection{Security Requirements}
\paragraph{HTTP Connections}
\begin{enumerate}

\item{NFRT-S1}\\
Type: Static, Manual, Code Inspection\\
Initial State: N/A\\
Input/Condition: N/A\\
Output/Result: All connections between the system and the APIs use HTTPS requests\\
How test will be performed: The test team will inspect the code and check if all connections between the system and APIs use HTTP requests
\end{enumerate}

\subsubsection{Area of Testing2}

...

\subsection{Traceability Between Test Cases and Requirements}
\begin{table}[H]
    \centering
    \caption{Traceability Matrix: Functional Requirement}
    \begin{tabular}{lcc}
        ~ & \textbf{FRT-??} & \textbf{FRT-??}\\
        \textbf{REQ?} & X & ~\\
        \textbf{REQ?} & ~ & X\\
    \end{tabular}
    \label{Traceability Matrix: Functional Requirement}
\end{table}

\begin{table}[H]
    \centering
    \caption{Traceability Matrix: Non-Functional Requirement}
    \begin{tabular}{lcc}
        ~ & \textbf{NFRT-P1} & \textbf{NFRT-S1}\\
        \textbf{P1} & X & ~\\
        \textbf{S1} & ~ & X\\
    \end{tabular}
    \label{Traceability Matrix: Non-Functional Requirement}
\end{table}

\section{Tests for Proof of Concept}

\subsection{Area of Testing1}
        
\paragraph{Title for Test}

\begin{enumerate}

\item{test-id1}\\
Type: Functional, Dynamic, Manual, Static etc.\\
Initial State: \\
Input: \\
Output: \\
How test will be performed: \\

\item{test-id2}\\
Type: Functional, Dynamic, Manual, Static etc.\\
Initial State: \\
Input: \\
Output: \\
How test will be performed: 

\end{enumerate}

\subsection{Area of Testing2}

...

    
\section{Comparison to Existing Implementation}
N/A

\section{Unit Testing Plan}
Unit testing will be performed through the use of the Pytest testing framework.

\subsection{Unit testing of internal functions}
Unit testing of internal functions will be performed for every function to ensure robustness. These tests will consist of a combination of partition testing and fuzz testing. Through this, verifying that the functions react in a predictable way. Following this, the functions will then undergo integration testing, to verify the compliance of the system with the SRS. During integration testing, stubs and drivers will be created as needed. For the testing of internal functions, the team will aim for a coverage of at minimum 85\%.

\subsection{Unit testing of output files}       
The only output file will be the application executable, which shall be tested for correctness as a whole as well as the fulfillment of the SRS. This testing will take the form of system testing and acceptance testing. The system testing will test the performance, the behavior under extreme/varying load, and scalability with the number of users. The acceptance testing will ensure the SRS has been fulfilled.

\bibliographystyle{plainnat}
\bibliography{SRS}
\newpage

\section{Appendix}
% This is where you can place additional information.

% \subsection{Symbolic Parameters}
% The definition of the test cases will call for SYMBOLIC\_CONSTANTS. Their values are defined in this section for easy maintenance.

% \subsection{Usability Survey Questions?}
% This is a section that would be appropriate for some teams.

\end{document}
