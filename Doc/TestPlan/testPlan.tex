\documentclass[12pt, titlepage]{article}

\usepackage{booktabs}
\usepackage{tabularx}
\usepackage{hyperref}
\hypersetup{
    colorlinks,
    citecolor=black,
    filecolor=black,
    linkcolor=red,
    urlcolor=blue
}
\usepackage[round]{natbib}
\usepackage{float}
\usepackage{setspace}
\usepackage{xspace}

% Document margins
\usepackage[left=1in,top=1in,right=1in,bottom=1in]{geometry}

\title{SE 3XA3: Test Plan\\ScrumBot}

\author{
    Team 304, ScrumBot
        \\ Arkin Modi, modia1
        \\ Leon So, sol4
        \\ Timothy Choy, choyt2
}
\date{Last Updated: February 27, 2020}

%\input{../Comments}

\begin{document}
\newpage
\maketitle

\pagenumbering{roman}
\tableofcontents
\listoftables
\listoffigures

\newpage

\begin{table}[!h]
    \caption{Revision History} \label{TblRevisionHistory}
    \begin{tabularx}{\textwidth}{llX}
        \toprule
            \textbf{Date} & \textbf{Developer(s)} & \textbf{Change}\\
        \midrule
            January 23, 2020 & Arkin Modi & Copy template\\
            February 20, 2020 & Arkin Modi & Created the Purpose Section\\
            February 27, 2020 & Timothy Choy, Leon So & Updated General Information section\\
            February 27, 2020 & Arkin Modi & Worked on Test Team, Automated Testing Approach, and Testing Tools\\
        \bottomrule
    \end{tabularx}
\end{table}


\newpage

\pagenumbering{arabic}

\section{General Information}

\subsection{Purpose}
The purpose of this document is to outline the testing, validation, and verification process of the functional and non-functional requirements, for the ScrumBot project. These test cases were conceived before the implementation and therefore will be used by the project members for future reference during the development process.

\subsection{Scope}
% The test plan outlines all tests which will be conducted during this project, and the testing tools that will be used.

\subsection{Acronyms, Abbreviations, and Symbols}
    
\begin{table}[hbp]
    \caption{Table of Abbreviations}
    \label{Table}
    \begin{tabularx}{\textwidth}{p{3cm}X}
        \toprule
        \textbf{Abbreviation} & \textbf{Definition} \\
        \midrule
        POC & Proof of Concept\\
        SRS & Software Requirements Specification\\
        \bottomrule
    \end{tabularx}
\end{table}

\begin{table}[!htbp]
    \caption{Table of Definitions}
    \label{Table}
    \begin{tabularx}{\textwidth}{p{3cm}X}
        \toprule
        \textbf{Term} & \textbf{Definition}\\
        \midrule
        Pytest & A unit testing framework for Python\\
        Term2 & Definition2\\
        \bottomrule
    \end{tabularx}
\end{table} 

\subsection{Overview of Document}
% How we are testing
This document outlines a test plan that fully encompasses all requirements of ScrumBot, as stated in the SRS. This document includes relevant information concerning: test team, automated testing, testing tools, testing schedule, unit-testing, and test cases.

\section{Plan}
\subsection{Software Description}
Scrum is an Agile process framework widely used in industry for managing and coordinating collaborative projects. Scrum being a process based on the agile development method, follows a highly iterative process and often has heavy customer involvement, therefore it can be often be complex. With Discord being a popular communication tool used by many teams of software developers today, ScrumBot provides a solution that directly integrates the management of a scrum development cycle into the communication channels. ScrumBot will allow for better management and organization of retrospectives, stand-ups, and other scrum/agile stages used by software teams within their routine communication channel.

\subsection{Test Team}
The test team will consist of all the members of the project: Arkin Modi, Leon So, and Timothy Choy.

\subsection{Automated Testing Approach}
Testing shall be automated with the use of the GitLab's CI/CD tool. The tests will be run every time a commit is pushed to the repository.

\subsection{Testing Tools}
The unit tests will be written using the Pytest framework. Static testing will be done with the use of Pylint.

\subsection{Testing Schedule}
See Gantt Chart at the following url, \url{https://gitlab.cas.mcmaster.ca/modia1/ScrumBot/-/blob/master/ProjectSchedule/}.

\section{System Test Description}
    
\subsection{Tests for Functional Requirements}

\subsubsection{Area of Testing1}
        
\paragraph{Title for Test}

\begin{enumerate}

\item{test-id1\\}

Type: Functional, Dynamic, Manual, Static etc.
                    
Initial State: 
                    
Input: 
                    
Output: 
                    
How test will be performed: 
                    
\item{test-id2\\}

Type: Functional, Dynamic, Manual, Static etc.
                    
Initial State: 
                    
Input: 
                    
Output: 
                    
How test will be performed: 

\end{enumerate}

\subsubsection{Area of Testing2}

...

\subsection{Tests for Nonfunctional Requirements}

\subsubsection{Area of Testing1}
        
\paragraph{Title for Test}

\begin{enumerate}

\item{test-id1\\}

Type: 
                    
Initial State: 
                    
Input/Condition: 
                    
Output/Result: 
                    
How test will be performed: 
                    
\item{test-id2\\}

Type: Functional, Dynamic, Manual, Static etc.
                    
Initial State: 
                    
Input: 
                    
Output: 
                    
How test will be performed: 

\end{enumerate}

\subsubsection{Area of Testing2}

...

\subsection{Traceability Between Test Cases and Requirements}

\section{Tests for Proof of Concept}

\subsection{Area of Testing1}
        
\paragraph{Title for Test}

\begin{enumerate}

\item{test-id1\\}

Type: Functional, Dynamic, Manual, Static etc.
                    
Initial State: 
                    
Input: 
                    
Output: 
                    
How test will be performed: 
                    
\item{test-id2\\}

Type: Functional, Dynamic, Manual, Static etc.
                    
Initial State: 
                    
Input: 
                    
Output: 
                    
How test will be performed: 

\end{enumerate}

\subsection{Area of Testing2}

...

    
\section{Comparison to Existing Implementation} 
                
\section{Unit Testing Plan}
        
\subsection{Unit testing of internal functions}
        
\subsection{Unit testing of output files}       

\bibliographystyle{plainnat}

\bibliography{SRS}

\newpage

\section{Appendix}

This is where you can place additional information.

\subsection{Symbolic Parameters}

The definition of the test cases will call for SYMBOLIC\_CONSTANTS.
Their values are defined in this section for easy maintenance.

\subsection{Usability Survey Questions?}

This is a section that would be appropriate for some teams.

\end{document}
