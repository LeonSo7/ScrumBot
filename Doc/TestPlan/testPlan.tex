\documentclass[12pt, titlepage]{article}

\usepackage{booktabs}
\usepackage{tabularx}
\usepackage{hyperref}
\hypersetup{
    colorlinks,
    citecolor=black,
    filecolor=black,
    linkcolor=red,
    urlcolor=blue
}
\usepackage[round]{natbib}
\usepackage{float}
\usepackage{setspace}
\usepackage{xspace}
\usepackage{pdflscape}
\usepackage{adjustbox}
\usepackage{booktabs}

% Different Coloured Text and Strikethrough
\usepackage{xcolor}
\usepackage{ulem}

% Document margins
\usepackage[left=1in,top=1in,right=1in,bottom=1in]{geometry}

\title{SE 3XA3: Test Plan\\ScrumBot}

\author{
	Team 304, ScrumBot
		\\ Arkin Modi, modia1
        \\ Leon So, sol4
        \\ Timothy Choy, choyt2
}
\date{Last Updated: April 6, 2020}

%\input{../Comments}

\begin{document}
\newpage
\maketitle

\pagenumbering{roman}
\tableofcontents
\listoftables
\listoffigures

\newpage

\begin{table}[!hbp]
    \caption{Revision History} \label{TblRevisionHistory}
    \begin{tabularx}{\textwidth}{llX}
        \toprule
            \textbf{Date} & \textbf{Developer(s)} & \textbf{Change}\\
        \midrule
            January 23, 2020 & Arkin Modi & Copy template\\
            February 20, 2020 & Arkin Modi & Created the Purpose Section\\
            February 27, 2020 & Timothy Choy & Worked on Scope and Acronyms and Abbreviations\\
            February 27, 2020 & Leon So & Worked on Scope and Overview of Document\\
            February 27, 2020 & Leon So & Worked on Software Description\\
            February 27, 2020 & Arkin Modi & Worked on Test Team, Automated Testing Approach, and Testing Tools\\
            February 27, 2020 & Leon So & Worked on Tests for NFRs\\
            February 27, 2020 & Arkin Modi & Worked on the Unit Testing Plan for internal functions and output files\\
            February 28, 2020 & Timothy Choy & Worked on the Proof of Concept Testing\\
            February 28, 2020 & Leon So & Worked on Tests for NFRs\\
            February 28, 2020 & Timothy Choy & Updated Definitions, Worked on Tests for FRs\\
            February 28, 2020 & Leon So & Worked on Usability Survey, Worked on Tests for FRs, Software Description\\
            February 28, 2020 & Arkin Modi & Worked on Traceability Matrices and Tests for FRs\\
            February 28, 2020 & Leon So, Timothy Choy & Proofread Document\\
            February 28, 2020 & Everyone & Final Revisions\\
            April 6, 2020 & Arkin Modi, Leon So & Revised all test cases\\
        \bottomrule
    \end{tabularx}
\end{table}


\newpage

\pagenumbering{arabic}

\section{General Information}

\subsection{Purpose}
The purpose of this document is to outline the testing, validation, and verification process of the functional and non-functional requirements, for the ScrumBot project. These test cases were conceived before the implementation and therefore will be used by the project members for future reference during the development process and testing process.

\subsection{Scope}
% The test plan outlines all tests which will be conducted during this project, and the testing tools that will be used.
% The scope of the test plan, what and how we are going to do this
This test plan will provide a method to fully test ScrumBot by performing tests both at a modular level using unit tests created through Pytest, as well as at higher level, through using exploratory testing and specification-based testing. The unit tests cases will also cover partition testing, fuzz testing, and boundary testing.

\subsection{Acronyms, Abbreviations, and Symbols}
	
\begin{table}[hbp]
    \caption{Table of Abbreviations}
    \label{Table}
    \begin{tabularx}{\textwidth}{p{3cm}X}
        \toprule
        \textbf{Abbreviation} & \textbf{Definition} \\
        \midrule
        CD & Continuous Delivery/Deployment\\
        CI & Continuous Integration\\
        EDT & Eastern Daylight Time (UTC-4)\\
        EST & Eastern Standard Time (UTC-5)\\
        FR & Functional Requirement\\
        HTTP & HyperText Transfer Protocol\\
        MVC & Model View Controller\\
        NFR & Non-functional Requirement\\
        POC & Proof of Concept\\
        SRS & Software Requirements Specification\\
        UTC & Coordinated Universal Time\\
        \bottomrule
    \end{tabularx}
\end{table}
\newpage
\begin{table}[H]
    \caption{Table of Definitions}
    \label{Table}
    \begin{tabularx}{\textwidth}{p{4cm}X}
        \toprule
        \textbf{Term} & \textbf{Definition}\\
        \midrule
        Acceptance Testing & A method of testing which is conducted to determine if the requirements of the specification are met\\
        Boundary Testing & A method of testing where values are chosen on semantically significant boundaries\\
        Business Analyst & Communicates and coordinates project requirements and deadlines between the Product Owner(s), Scrum Master, and Development Team\\
        Checklist & A method of testing through inspecting the code\\
        Code Inspection & A method of static testing where developers walk through the code\\
        Discord & A chat application. The platform in which ScrumBot will be implemented.\\
        Dynamic Testing & A method of testing where code is executed\\
        Exploratory Testing & A method of testing where the tester simultaneously learns the code while testing it. It approaches testing from a user's viewpoint\\
        Fuzz Testing & A method of testing where random inputs are given to attempt to violate assertions\\
        Grooming & A meeting where the Business Analyst communicates and coordinates project requirements and deadlines with the Scrum Master and Development Team\\
        Integration Testing & A method of testing where individual software modules are combined and tested as a group\\
        Kanban Board & A method of scheduling tasks through categorizing tasks to improve efficiency\\
        Partition Testing & A method of testing where the input domain is partitioned and input values are selected from the partitions\\
        Pylint & A Python linter, used for static testing\\
        Pytest & A unit testing framework for Python\\
        Retrospective & A team meeting for reflecting on an Scrum sprint\\
        ScrumBot & The Discord bot in development\\
        Scrum Master & The facilitator for an agile development team who plans, leads and organizes Scrum meetings\\
        Specification-based Testing & A method of testing where test cases are built based on the requirements specification\\
        \bottomrule
    \end{tabularx}
\end{table}	

\begin{table}[]
    \centering
    \begin{tabularx}{\textwidth}{p{4cm}X}
        \toprule
        \textbf{Term} & \textbf{Definition}\\
        \midrule
        Sprint & A set time period where specific work has to be completed and made ready for review\\
        Stand-up & A daily coordination meeting used in the Scrum framework\\
        Static Testing & A method of testing where code is not executed\\
        System Testing & A method of testing where the tests are performed on the system as a whole\\
        Trello & A web based Kanban project management system\\
        Unit Testing & A method of testing focused on testing individual methods and functions\\
        \bottomrule
    \end{tabularx}
\end{table}

\newpage

\subsection{Overview of Document}
% The overview of this document
This document outlines a test plan that fully encompasses all requirements of ScrumBot specified in the SRS. This document includes relevant information concerning: test team, automated testing, testing tools, testing schedule, unit-testing, and test cases.

\section{Plan}
\subsection{Software Description}
Scrum is an Agile process framework widely used in industry for managing and coordinating collaborative projects. Scrum being a process based on the agile development method, follows a highly iterative process and often has heavy customer involvement, therefore it can be often be complex. With Discord being a popular communication tool used by many teams of software developers today, ScrumBot provides a solution that directly integrates the management of a scrum development cycle into the communication channels. ScrumBot will allow for better management and organization of retrospectives, stand-ups, and other scrum/agile stages used by software teams within their routine communication channel. ScrumBot will provide features to add and manage Scrum meetings, as well as to store information relevant to those meetings. ScrumBot will also allow Scrum roles to be assigned to members of the Discord channel.

\subsection{Test Team}
The test team will consist of all the members of the project: Arkin Modi, Leon So, and Timothy Choy.

\subsection{Automated Testing Approach}
Testing shall be \textcolor{red}{partially} automated with the use of \sout{the GitLab's CI/CD tool and }the Pytest framework. The tests will be run every time a commit is pushed to the repository.

\subsection{Testing Tools}
The unit tests will be written using the Pytest framework. Static testing will be done with the use of Pylint. \textcolor{red}{Manual testing will be done by the test team.}

\subsection{Testing Schedule}
See Gantt Chart at the following URL, \url{https://gitlab.cas.mcmaster.ca/modia1/ScrumBot/-/blob/master/ProjectSchedule/}.

\section{System Test Description}
\subsection{Tests for Functional Requirements}
\textcolor{red}{\textbf{All previous test cases have been changed.}}

\subsubsection{Installation}
\begin{enumerate}
    \item{\textbf{FRT-BE1}}\\
    Type: Functional, Dynamic, Manual\\
    Initial State: A Discord channel is active\\
    Input: ScrumBot is added into the channel\\
    Output: ScrumBot welcome message\\
    How test will be performed: ScrumBot will be manually added to a Discord channel.\\
\end{enumerate}

\subsubsection{Project Creation}
\begin{enumerate}
    \item{\textbf{FRT-BE2-1}}\\
    Type: Functional, Dynamic, Unit, Automated\\
    Initial State: Empty project list\\
    Input: Create New project, with description\\
    Output: New project created and added to list\\
    How test will be performed: Pytest will run the internal commands that ScrumBot will use to create a project.

    \item{\textbf{FRT-BE2-2}}\\
    Type: Functional, Dynamic, Unit, Automated\\
    Initial State: Empty project list\\
    Input: Create new project, without description\\
    Output: New project created and added to list\\
    How test will be performed: Pytest will run the internal commands that ScrumBot will use to create a project.
\end{enumerate}

\subsubsection{Project Removal}
\begin{enumerate}
    \item{\textbf{FRT-BE3-1}}\\
    Type: Functional, Dynamic, Unit, Automated\\
    Initial State: Project list containing one project with ID 0\\
    Input: Remove project by ID, ID is 0\\
    Output: Project list empty, project with ID 0 is removed\\
    How test will be performed: Pytest will run the internal commands that ScrumBot will use to remove a project.

    \item{\textbf{FRT-BE3-2}}\\
    Type: Functional, Dynamic, Unit, Automated\\
    Initial State: Empty project list\\
    Input: Remove project by ID\\
    Output: KeyError\\
    How test will be performed: Pytest will run the internal commands that ScrumBot will use to remove a project.
\end{enumerate}

\subsubsection{Sprint-Planning Meeting}
\begin{enumerate}
    \item{\textbf{FRT-BE4-1}}\\
    Type: Functional, Dynamic, Unit, Automated\\
    Initial State: Project with a sprint and no tasks\\
    Input: Add a task without details\\
    Output: Task added to sprint\\
    How test will be performed: Pytest will run the internal commands that ScrumBot will use to add task to a sprint.

    \item{\textbf{FRT-BE4-2}}\\
    Type: Functional, Dynamic, Unit, Automated\\
    Initial State: Project with a sprint and no tasks\\
    Input: Add a task with details\\
    Output: Task added to sprint\\
    How test will be performed: Pytest will run the internal commands that ScrumBot will use to add task to a sprint.
    
    \item{\textbf{FRT-BE4-3}}\\
    Type: Functional, Dynamic, Unit, Automated\\
    Initial State: Project with no sprints and no tasks \\
    Input: Add a task\\
    Output: IndexError\\
    How test will be performed: Pytest will run the internal commands that ScrumBot will use to add task to a sprint.
    
    \item{\textbf{FRT-BE4-4}}\\
    Type: Functional, Dynamic, Unit, Automated\\
    Initial State: Project with a sprint and tasks \\
    Input: Get all tasks by sprint IDs\\
    Output: A list of tasks\\
    How test will be performed: Pytest will run the internal commands that ScrumBot will use to get all tasks from sprint.
    
    \item{\textbf{FRT-BE4-5}}\\
    Type: Functional, Dynamic, Unit, Automated\\
    Initial State: Project with sprints and tasks, contains task with ID 0 \\
    Input: Get a single task with ID 0\\
    Output: Task with ID 0\\
    How test will be performed: Pytest will run the internal commands that ScrumBot will use to get a single task from sprint.
    
    \item{\textbf{FRT-BE4-6}}\\
    Type: Functional, Dynamic, Unit, Automated\\
    Initial State: Project no sprints and no tasks\\
    Input: Get a single task with ID 0\\
    Output: IndexError\\
    How test will be performed: Pytest will run the internal commands that ScrumBot will use to get a single task from a sprint.
    
    \item{\textbf{FRT-BE4-7}}\\
    Type: Functional, Dynamic, Unit, Automated\\
    Initial State: Project no sprints and no tasks\\
    Input: Get a list of tasks\\
    Output: IndexError\\
    How test will be performed: Pytest will run the internal commands that ScrumBot will use to get a list of tasks from a sprint.
    
    \item{\textbf{FRT-BE4-8}}\\
    Type: Functional, Dynamic, Unit, Automated\\
    Initial State: Project with a sprint and a task\\
    Input: Add feedback to a task by task ID\\
    Output: Feedback added to task of last sprint\\
    How test will be performed: Pytest will run the internal commands that ScrumBot will use to add feedback to a task in the last sprint.
 
     
    \item{\textbf{FRT-BE4-9}}\\
    Type: Functional, Dynamic, Unit, Automated\\
    Initial State: Project with no sprint and no task\\
    Input: Add feedback to a task by task ID\\
    Output: Index Error\\
    How test will be performed: Pytest will run the internal commands that ScrumBot will use to add feedback to a task in the last sprint.
    
    \item{\textbf{FRT-BE4-10}}\\
    Type: Functional, Dynamic, Unit, Automated\\
    Initial State: Project a sprint, a task, and a feedback\\
    Input: Get feedback from task using sprint ID and task ID\\
    Output: Feedback of sprint with inputted sprint ID and task with inputted task ID\\
    How test will be performed: Pytest will run the internal commands that ScrumBot will use to get feedback from task and sprint.
    
    \item{\textbf{FRT-BE4-11}}\\
    Type: Functional, Dynamic, Unit, Automated\\
    Initial State: Project with no sprint, no task\\
    Input: Get feedback from task using sprint ID and task ID\\
    Output: IndexError\\
    How test will be performed: Pytest will run the internal commands that ScrumBot will use to get feedback from task and sprint.
\end{enumerate}

\subsubsection{Stand-up Meeting}
\begin{enumerate}
    \item{\textbf{FRT-BE5-1}}\\
    Type: Functional, Dynamic, Unit, Automated\\
    Initial State: Project with a sprint and a task\\
    Input: Remove task by task ID\\
    Output: Task removed from last sprint\\
    How test will be performed: Pytest will run the internal commands that ScrumBot will use to remove a task from the last sprint.
    
    \item{\textbf{FRT-BE5-2}}\\
    Type: Functional, Dynamic, Unit, Automated\\
    Initial State: Project with no sprint\\
    Input: Remove task by task ID\\
    Output: IndexError\\
    How test will be performed: Pytest will run the internal commands that ScrumBot will use to remove a task from the last sprint.
    
    \item{\textbf{FRT-BE5-3}}\\
    Type: Functional, Dynamic, Unit, Automated\\
    Initial State: Project with a sprint and no tasks\\
    Input: Remove task by task ID\\
    Output: KeyError\\
    How test will be performed: Pytest will run the internal commands that ScrumBot will use to remove a task from the last sprint.
    
    \item{\textbf{FRT-BE5-4}}\\
    Type: Functional, Dynamic, Unit, Automated\\
    Initial State: Project with a sprint and a task with ID 0\\
    Input: Set details of task with ID 0\\
    Output: Task with ID 0 updates with new details\\
    How test will be performed: Pytest will run the internal commands that ScrumBot will use to set description of a task from the last sprint.
    
    \item{\textbf{FRT-BE5-5}}\\
    Type: Functional, Dynamic, Unit, Automated\\
    Initial State: Project with no sprints\\
    Input: Set details of task with ID 0\\
    Output: IndexError\\
    How test will be performed: Pytest will run the internal commands that ScrumBot will use to set description of a task from the last sprint.
\end{enumerate}

\subsubsection{Retrospective Meeting}
\begin{enumerate}
    \item{\textbf{FRT-BE6-1}}\\
    Type: Functional, Dynamic, Unit, Automated\\
    Initial State: Project with a sprint, a task and feedback\\
    Input: Remove feedback by sprint and task IDs\\
    Output: Feedback removed\\
    How test will be performed: Pytest will run the internal commands that ScrumBot will use to remove feedback from specified sprint and task.
    
    \item{\textbf{FRT-BE6-2}}\\
    Type: Functional, Dynamic, Unit, Automated\\
    Initial State: Project with no sprint\\
    Input: Remove feedback by sprint and task IDs\\
    Output: IndexError\\
    How test will be performed: Pytest will run the internal commands that ScrumBot will use to remove feedback from specified sprint and task.
\end{enumerate}

\subsubsection{Grooming Meeting}
\begin{enumerate}
    \item{\textbf{FRT-BE7-1}}\\
    Type: Functional, Dynamic, Unit, Automated\\
    Initial State: Project with no requirements\\
    Input: Add requirement\\
    Output: Requirement added\\
    How test will be performed: Pytest will run the internal commands that ScrumBot will use to add a requirement to the project.
    
    \item{\textbf{FRT-BE7-2}}\\
    Type: Functional, Dynamic, Unit, Automated\\
    Initial State: Project with requirements\\
    Input: Get a list of requirement\\
    Output: A list of requirements\\
    How test will be performed: Pytest will run the internal commands that ScrumBot will use to get a list of requirements to the project.
    
    \item{\textbf{FRT-BE7-3}}\\
    Type: Functional, Dynamic, Unit, Automated\\
    Initial State: Project with requirements\\
    Input: Remove a requirement by requirement ID\\
    Output: Requirement removed\\
    How test will be performed: Pytest will run the internal commands that ScrumBot will use to remove a requirement from the project.
    
    \item{\textbf{FRT-BE7-4}}\\
    Type: Functional, Dynamic, Unit, Automated\\
    Initial State: Project with no requirements\\
    Input: Remove a requirement by requirement ID\\
    Output: IndexError\\
    How test will be performed: Pytest will run the internal commands that ScrumBot will use to remove a requirement from the project.
\end{enumerate}

\subsubsection{Add a Meeting}
\begin{enumerate}
    \item{\textbf{FRT-BE8-1}}\\
    Type: Functional, Dynamic, Unit, Automated\\
    Initial State: Project with no meetings\\
    Input: Add a meeting with a description\\
    Output: Meeting added\\
    How test will be performed: Pytest will run the internal commands that ScrumBot will use to add a meeting to the project.
    
    \item{\textbf{FRT-BE8-2}}\\
    Type: Functional, Dynamic, Unit, Automated\\
    Initial State: Project with no meetings\\
    Input: Add a meeting with no description\\
    Output: Meeting added\\
    How test will be performed: Pytest will run the internal commands that ScrumBot will use to add a meeting to the project.
    
    \item{\textbf{FRT-BE8-3}}\\
    Type: Functional, Dynamic, Unit, Automated\\
    Initial State: Project with meetings\\
    Input: Get a meeting by meeting ID\\
    Output: Meeting with specified ID\\
    How test will be performed: Pytest will run the internal commands that ScrumBot will use to get a meeting from the project.
    
    \item{\textbf{FRT-BE8-4}}\\
    Type: Functional, Dynamic, Unit, Automated\\
    Initial State: Project with meetings that contains meeting with ID 0\\
    Input: Get meeting name with ID 0\\
    Output: Meeting name of meeting with ID 0\\
    How test will be performed: Pytest will run the internal commands that ScrumBot will use to get the name of a meeting from the project.
    
    \item{\textbf{FRT-BE8-5}}\\
    Type: Functional, Dynamic, Unit, Automated\\
    Initial State: Project with meetings that contains meeting with ID 0\\
    Input: Get meeting description with ID 0\\
    Output: Meeting description of meeting with ID 0\\
    How test will be performed: Pytest will run the internal commands that ScrumBot will use to get the description of a meeting from the project.
    
    \item{\textbf{FRT-BE8-6}}\\
    Type: Functional, Dynamic, Unit, Automated\\
    Initial State: Project with meetings that contains meeting with ID 0 and has no description\\
    Input: Get meeting description with ID 0\\
    Output: Meeting description of meeting with ID 0\\
    How test will be performed: Pytest will run the internal commands that ScrumBot will use to get the description of a meeting from the project.
    
    \item{\textbf{FRT-BE8-7}}\\
    Type: Functional, Dynamic, Unit, Automated\\
    Initial State: Project with meetings that contains meeting with ID 0 and has a description\\
    Input: Get meeting description with ID 0\\
    Output: "No description"\\
    How test will be performed: Pytest will run the internal commands that ScrumBot will use to get the description of a meeting from the project.
    
    \item{\textbf{FRT-BE8-8}}\\
    Type: Functional, Dynamic, Unit, Automated\\
    Initial State: Project with no meetings\\
    Input: Get meeting name with ID 0\\
    Output: KeyError\\
    How test will be performed: Pytest will run the internal commands that ScrumBot will use to get the name of a meeting from the project.
    
    \item{\textbf{FRT-BE8-8}}\\
    Type: Functional, Dynamic, Unit, Automated\\
    Initial State: Project with no meetings\\
    Input: Get meeting description with ID 0\\
    Output: KeyError\\
    How test will be performed: Pytest will run the internal commands that ScrumBot will use to get the description of a meeting from the project.
    
    \item{\textbf{FRT-BE8-9}}\\
    Type: Functional, Dynamic, Unit, Automated\\
    Initial State: Project with no meetings\\
    Input: Add a meeting of every meeting type (4 meetings in total)\\
    Output: Added meeting\\
    How test will be performed: Pytest will run the internal commands that ScrumBot will use to add a meeting to the project.
    
    \item{\textbf{FRT-BE8-10}}\\
    Type: Functional, Dynamic, Unit, Automated\\
    Initial State: Project with no meetings\\
    Input: Add a meeting of an invalid meeting type\\
    Output: TypeError\\
    How test will be performed: Pytest will run the internal commands that ScrumBot will use to add a meeting to the project.
\end{enumerate}

\subsubsection{Cancel a Meeting}

\begin{enumerate}
    \item{\textbf{FRT-BE9-1}}\\
    Type: Functional, Dynamic, Unit, Automated\\
    Initial State: Project with meetings and contains meeting with ID 0\\
    Input: Remove meeting with ID 0\\
    Output: Meeting with ID 0 removed\\
    How test will be performed: Pytest will run the internal commands that ScrumBot will use to remove a meeting from the project.
    
    \item{\textbf{FRT-BE9-2}}\\
    Type: Functional, Dynamic, Unit, Automated\\
    Initial State: Project with meetings and contains no meetings with ID 0\\
    Input: Remove meeting with ID 0\\
    Output: KeyError\\
    How test will be performed: Pytest will run the internal commands that ScrumBot will use to remove a meeting from the project.
\end{enumerate}

\subsubsection{List Scheduled Meetings}
\begin{enumerate}
    \item{\textbf{FRT-BE10-1}}\\
    Type: Functional, Dynamic, Unit, Automated\\
    Initial State: Project with four meetings, each meeting of a different type\\
    Input: Get a list of all meetings\\
    Output: A list of all meetings\\
    How test will be performed: Pytest will run the internal commands that ScrumBot will use to get a list of meetings from the project.
\end{enumerate}

\subsection{Tests for Non-Functional Requirements}
\textcolor{red}{\textbf{All previous test cases have been changed.}}

\subsubsection{Look and Feel Requirements}
\paragraph{Appearance Requirements}
\begin{enumerate}
    \item{\textbf{NFRT-LF-1}}\\
    Type: Manual, Dynamic, Checklist\\
    Initial State: ScrumBot active in Discord channel\\
    Input: Set of all ScrumBot commands entered\\
    Output: ScrumBot outputs text formatted according to the text format of ScrumBot\\
    How test will be performed: The test team will run commands in the Discord chat with ScrumBot active and check if the text format is appropriate and in accordance with Discord's text format
\end{enumerate}

\paragraph{Style Requirements}
\begin{enumerate}
    \item{\textbf{NFRT-LF-2}}\\
    Type: Manual, Usability Survey\\
    How test will be performed: The test team will ask a sample set of users to answer questions on a usability survey after using ScrumBot for the first time. A sample survey is included in the appendix. Questions will be asked regarding their opinion on the clarity of role names, and how these roles are represented on Discord. A sample survey is included in the appendix.
    
    \item{\textbf{NFRT-LF-3}}\\
    Type: Manual, Dynamic, Checklist\\
    Initial State: ScrumBot active in Discord channel\\
    Input: One of each role available is added\\
    Output: ScrumBot adds the roles and permissions to the assigned users. The role names  should be colour coded.\\
    How test will be performed: The test team will add roles in the Discord channel. One of each role available will be assigned.
\end{enumerate}

\subsubsection{Usability and Humanity Requirements}
\paragraph{Ease of Use \& Personalization and Internationalization Requirements}
\begin{enumerate}
    \item{\textbf{NFRT-UH-1}}\\
    Type: Manual, Usability Survey\\
    How test will be performed: The test team will ask a sample set of users to answer questions on a usability survey after using ScrumBot for the first time. A sample survey is included in the appendix.
\end{enumerate}

\paragraph{Learning Requirements}
\begin{enumerate}
    \item{\textbf{NFRT-UH-2}}\\
    Type: Dynamic, Manual\\
    Initial State: ScrumBot active in Discord channel\\
    Input: Help command entered\\
    Output: ScrumBot should output help menu\\
    How test will be performed: The test team will enter the help command into Discord chat with ScrumBot active
\end{enumerate}

\subsubsection{Performance Requirements}
\paragraph{Response Speed}
\begin{enumerate}
\item{\textbf{NFRT-P-1}}\\
    Type: Dynamic, Manual\\
    Initial State: No commands being made\\
    Input: Command entered\\
    Output: Response should be received within 2 seconds of the input being sent\\
    How test will be performed: The test team will enter a command into the Discord channel with ScrumBot active. ScrumBot should provide a response within 2ms of the command being entered. A helper method will record the time and report to the tester the total time taken.
\end{enumerate}

\subsubsection{Operational and Environmental Requirements}
\paragraph{Expected Environment}
\begin{enumerate}
    \item{\textbf{NFRT-OE-1}}\\
    Type: Dynamic, Manual\\
    Initial State: New Discord channel without ScrumBot\\
    Input/Condition: Add ScrumBot to the Discord channel\\
    Output/Result: ScrumBot should be active in the Discord channel once added\\
    How test will be performed: The test team will follow the provided installation documentation and add ScrumBot to a brand new Discord channel
\end{enumerate}

\paragraph{Requirements for Interfacing with Adjacent Systems}


\paragraph{Installability Requirements}
\begin{itemize}
    \item{\textbf{NFRT-OE-2}}\\
    Type: Dynamic, Manual\\
    Initial State: ScrumBot not yet installed on Discord server\\
    Input/Condition: Establish connection between Scrumbot and Discord\\
    Output/Result: Scrumbot should connect to Discord\\
    How test will be performed: The test team will follow documentation provided by Discord to add ScrumBot to the Discord server
\end{itemize}

\subsubsection{Maintainability and Support Requirements}
\paragraph{Maintainability Requirements}
\begin{enumerate}

\item{\textbf{NFRT-MS1}}\\
Type: Static, Manual, Code Inspection\\
Initial State: N/A\\
Input/Condition: N/A\\
Output/Result: The code is well documented with comments\\
How test will be performed: The test team will inspect the code and check if the code is adequately documented using comments

\item{\textbf{NFRT-MS2}}\\
Type: Static, Manual\\
Initial State: No Doxygen documents generated yet\\
Input/Condition: Generate Doxygen HTML and pdf\\
Output/Result: The Doxygen documentation is successfully compiled and generated\\
How test will be performed: The test team will try to generate Doxygen HTML and pdf

\item{\textbf{NFRT-MS3}}\\
Type: Static, Manual, Code Inspection\\
Initial State: N/A\\
Input/Condition: N/A\\
Output/Result: The code documentation is easy to understand\\
How test will be performed: The test team will review the code documentation and make sure it is easily understandable. The test team will survey other developers and verify that they too can easily understand the documentation
\end{enumerate}

\paragraph{Supportability Requirements}
\begin{enumerate}
\item{\textbf{NFRT-MS4-1}}\\
Type: Dynamic, Manual\\
Initial State: ScrumBot is active on the Discord channel\\
Input/Condition: User wants to open help menu\\
Output/Result: Help menu is displayed in the Discord chat\\
How test will be performed: The test team will attempt to open the help menu in the Discord chat

\end{enumerate}

\paragraph{Longevity Requirements}
\begin{enumerate}
    \item{\textbf{NFRT-MS4-2}}\\
    Type: Static, Manual, Code Inspection\\
    Initial State: N/A\\
    Input/Condition: N/A\\
    Output/Result: System is modularized into classes\\
    How test will be performed: The test team will perform a code inspection to ensure that the system is separated into modules
\end{enumerate}

\subsubsection{Security Requirements}
\paragraph{HTTP Connections}
\begin{enumerate}
    \item{\textbf{NFRT-S1}}\\
    Type: Static, Manual, Code Inspection\\
    Initial State: N/A\\
    Input/Condition: N/A\\
    Output/Result: All connections between the system and the APIs use HTTPS requests\\
    How test will be performed: The test team will inspect the code and check if all connections between the system and APIs use HTTP requests
\end{enumerate}

\newpage

\subsection{Traceability Between Test Cases and Requirements}
\noindent Test IDs correspond to the tests found in this document. Requirement IDs correspond to the requirements found in the SRS. \\
\textcolor{red}{\textbf{Tables were updated to reflect new test cases.}}

\begin{table}[H]
    \centering
    \caption{Traceability Matrix: Functional Requirement}
    \begin{adjustbox}{max width=0.7\paperwidth}
    \begin{tabular}{l|ccccccccccc}
        \textbf{Test IDs} & \multicolumn{11}{c}{\textbf{Requirement IDs}}\\
        \hline
        ~ & \textbf{BE1} & \textbf{BE2} & \textbf{BE3} & \textbf{BE4} & \textbf{BE5} & \textbf{BE6} & \textbf{BE7} & \textbf{BE8} & \textbf{BE9} & \textbf{BE10} & \textbf{BE11}\\
        \textbf{FRT-BE1}    & X & ~ & ~ & ~ & ~ & ~ & ~ & ~ & ~ & ~ & ~\\
        \textbf{FRT-BE2-1}  & ~ & X & ~ & ~ & ~ & ~ & ~ & ~ & ~ & ~ & ~\\
        \textbf{FRT-BE2-2}  & ~ & X & ~ & ~ & ~ & ~ & ~ & ~ & ~ & ~ & ~\\
        \textbf{FRT-BE3-1}  & ~ & ~ & X & ~ & ~ & ~ & ~ & ~ & ~ & ~ & ~\\
        \textbf{FRT-BE3-2}  & ~ & ~ & X & ~ & ~ & ~ & ~ & ~ & ~ & ~ & ~\\
        \textbf{FRT-BE4-1}  & ~ & ~ & ~ & X & ~ & ~ & ~ & ~ & ~ & ~ & ~\\
        \textbf{FRT-BE4-2}  & ~ & ~ & ~ & X & ~ & ~ & ~ & ~ & ~ & ~ & ~\\
        \textbf{FRT-BE4-3}  & ~ & ~ & ~ & X & ~ & ~ & ~ & ~ & ~ & ~ & ~\\
        \textbf{FRT-BE4-4}  & ~ & ~ & ~ & X & ~ & ~ & ~ & ~ & ~ & ~ & X\\
        \textbf{FRT-BE4-5}  & ~ & ~ & ~ & X & ~ & ~ & ~ & ~ & ~ & ~ & ~\\
        \textbf{FRT-BE4-6}  & ~ & ~ & ~ & X & ~ & ~ & ~ & ~ & ~ & ~ & ~\\
        \textbf{FRT-BE4-7}  & ~ & ~ & ~ & X & ~ & ~ & ~ & ~ & ~ & ~ & ~\\
        \textbf{FRT-BE4-8}  & ~ & ~ & ~ & X & ~ & ~ & ~ & ~ & ~ & ~ & ~\\
        \textbf{FRT-BE4-9}  & ~ & ~ & ~ & X & ~ & ~ & ~ & ~ & ~ & ~ & ~\\
        \textbf{FRT-BE4-10} & ~ & ~ & ~ & X & ~ & ~ & ~ & ~ & ~ & ~ & ~\\
        \textbf{FRT-BE4-11} & ~ & ~ & ~ & X & ~ & ~ & ~ & ~ & ~ & ~ & ~\\
        \textbf{FRT-BE5-1}  & ~ & ~ & ~ & ~ & X & ~ & ~ & ~ & ~ & ~ & ~\\
        \textbf{FRT-BE5-2}  & ~ & ~ & ~ & ~ & X & ~ & ~ & ~ & ~ & ~ & ~\\
        \textbf{FRT-BE5-3}  & ~ & ~ & ~ & ~ & X & ~ & ~ & ~ & ~ & ~ & ~\\
        \textbf{FRT-BE5-4}  & ~ & ~ & ~ & ~ & X & ~ & ~ & ~ & ~ & ~ & ~\\
        \textbf{FRT-BE5-5}  & ~ & ~ & ~ & ~ & X & ~ & ~ & ~ & ~ & ~ & ~\\
        \textbf{FRT-BE6-1}  & ~ & ~ & ~ & ~ & ~ & X & ~ & ~ & ~ & ~ & ~\\
        \textbf{FRT-BE6-2}  & ~ & ~ & ~ & ~ & ~ & X & ~ & ~ & ~ & ~ & ~\\
        \textbf{FRT-BE7-1}  & ~ & ~ & ~ & ~ & ~ & ~ & X & ~ & ~ & ~ & ~\\
        \textbf{FRT-BE7-2}  & ~ & ~ & ~ & ~ & ~ & ~ & X & ~ & ~ & ~ & ~\\
        \textbf{FRT-BE7-3}  & ~ & ~ & ~ & ~ & ~ & ~ & X & ~ & ~ & ~ & ~\\
        \textbf{FRT-BE7-4}  & ~ & ~ & ~ & ~ & ~ & ~ & X & ~ & ~ & ~ & ~\\
        \textbf{FRT-BE8-1}  & ~ & ~ & ~ & ~ & ~ & ~ & ~ & X & ~ & ~ & ~\\
        \textbf{FRT-BE8-2}  & ~ & ~ & ~ & ~ & ~ & ~ & ~ & X & ~ & ~ & ~\\
        \textbf{FRT-BE8-3}  & ~ & ~ & ~ & ~ & ~ & ~ & ~ & X & ~ & ~ & ~\\
        \textbf{FRT-BE8-4}  & ~ & ~ & ~ & ~ & ~ & ~ & ~ & X & ~ & ~ & ~\\
        \textbf{FRT-BE8-5}  & ~ & ~ & ~ & ~ & ~ & ~ & ~ & X & ~ & ~ & ~\\
        \textbf{FRT-BE8-6}  & ~ & ~ & ~ & ~ & ~ & ~ & ~ & X & ~ & ~ & ~\\
        \textbf{FRT-BE8-7}  & ~ & ~ & ~ & ~ & ~ & ~ & ~ & X & ~ & ~ & ~\\
        \textbf{FRT-BE8-8}  & ~ & ~ & ~ & ~ & ~ & ~ & ~ & X & ~ & ~ & ~\\
        \textbf{FRT-BE8-9}  & ~ & ~ & ~ & ~ & ~ & ~ & ~ & X & ~ & ~ & ~\\
        \textbf{FRT-BE8-10} & ~ & ~ & ~ & ~ & ~ & ~ & ~ & X & ~ & ~ & ~\\
        \textbf{FRT-BE9-1}  & ~ & ~ & ~ & ~ & ~ & ~ & ~ & ~ & X & ~ & ~\\
        \textbf{FRT-BE9-2}  & ~ & ~ & ~ & ~ & ~ & ~ & ~ & ~ & X & ~ & ~\\
        \textbf{FRT-BE10-1} & ~ & ~ & ~ & ~ & ~ & ~ & ~ & ~ & ~ & X & ~\\
    \end{tabular}
    \end{adjustbox}
    \label{Traceability Matrix: Functional Requirement}
\end{table}

\newpage
\begin{landscape}
\begin{table}[H]
    \centering
    \caption{Traceability Matrix: Non-Functional Requirement}
    \begin{adjustbox}{width=\paperwidth}
    \begin{tabular}{l|cccccccccccccccccccccc}
        \textbf{Test IDs} & \multicolumn{19}{c}{\textbf{Requirement IDs}}\\
        \hline
        ~ & \textbf{LF1} & \textbf{LF2} & \textbf{LF3} & \textbf{UH1} & \textbf{UH2} & \textbf{UH3} & \textbf{UH4} & \textbf{P1} & \textbf{P2} & \textbf{P3} & \textbf{OE1} & \textbf{OE2} & \textbf{OE3} & \textbf{OE4} & \textbf{OE5} & \textbf{OE6} & \textbf{OE7} & \textbf{MS1} & \textbf{MS2} & \textbf{MS3} & \textbf{MS4} & \textbf{S1}\\
        \textbf{NFRT-LF-1}    & X & ~ & ~ & ~ & ~ & ~ & ~ & ~ & ~ & ~ & ~ & ~ & ~ & ~ & ~ & ~ & ~ & ~ & ~ & ~ & ~ & ~\\
        \textbf{NFRT-LF-2}    & ~ & X & ~ & ~ & ~ & ~ & ~ & ~ & ~ & ~ & ~ & ~ & ~ & ~ & ~ & ~ & ~ & ~ & ~ & ~ & ~ & ~\\
    \end{tabular}
    \end{adjustbox}
    \label{Traceability Matrix: Non-Functional Requirement}
\end{table}
\end{landscape}

\section{Tests for Proof of Concept}

\subsection{Issues and Conflicts}
The proof of concept demonstration for ScrumBot consisted of a simple Python discord bot performing basic front-end tasks such as creating, listing, and deleting meeting times. There was no database connected to the proof of concept, so memory was not stored from instance to instance.\\
\noindent Issues that were found with the proof of concept were:
\begin{enumerate}
    \item The lack of a priority list for features
    \item The lack of a defined software architectural style
    \item The need for time zones, as people could be connecting to meetings from around the world
\end{enumerate}

\subsection{Resolution}
To resolve these issues, we have taken the list of business events from our SRS and have assigned priority to the events. In the case where ScrumBot will not be able to fulfill all the requirements in the given timeframe, ScrumBot will still be able to function as the prioritized functionalities will be implemented.\\ \\
\noindent In regards to the chosen architectural style, we have decided on using MVC as our primary architectural style. This best suits ScrumBot as the view module will be all the input and output from Discord, our controller module will be the commands run, and our model module will be the databases containing all the meeting information.\\\\
\noindent To tackle the issue regarding time zones, we plan on writing our times in EST or EDT, based on the date scheduled for the meeting. The choice is simply because of our current location, and makes it simpler for us to test and create meetings. However, a new feature we plan on implementing is the ability to convert between time zones given a meeting.
	
\section{Comparison to Existing Implementation}
N/A

\section{Unit Testing Plan}
Unit testing will be performed through the use of the Pytest testing framework.

\subsection{Unit testing of internal functions}
Unit testing of internal functions will be performed for every function to ensure robustness. These tests will consist of a combination of partition testing and fuzz testing. Through this, verifying that the functions react in a predictable way. Following this, the functions will then undergo integration testing, to verify the compliance of the system with the SRS. 
\\\\
\noindent During integration testing, stubs and drivers will be created as needed. For the testing of internal functions, the team will aim for a coverage of at minimum 85\%.

\subsection{Unit testing of output files}		
The only output file will be the application executable, which shall be tested for correctness as a whole as well as the fulfillment of the SRS. This testing will take the form of system testing and acceptance testing. The system testing will test the performance, the behavior under extreme/varying load, and scalability with the number of users. The acceptance testing will ensure the SRS has been fulfilled.

\bibliographystyle{plainnat}
\bibliography{SRS}
\newpage

\section{Appendix}
% This is where you can place additional information.

% \subsection{Symbolic Parameters}
% The definition of the test cases will call for SYMBOLIC\_CONSTANTS. Their values are defined in this section for easy maintenance.

\subsection{Usability Survey Questions}
These survey questions will be used to test usability requirements.
\begin{enumerate}
    \item Are you 13+ years old?
    \item (On a scale from 1-10) How intuitive did you find ScrumBot's commands?
    \item Did you find ScrumBot commands unnecessarily complex? If so, explain.
    \item Did you find ScrumBot easy to use?
    \item Would you use ScrumBot on a regular basis?
    \item Do you agree that the language used by ScrumBot is appropriate for English speakers?
    \item Are the role names clear and understandable?
    \item Is the system and user commands intuitive and easy to use?
\end{enumerate}
% This is a section that would be appropriate for some teams.

\end{document}
