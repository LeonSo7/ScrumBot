\documentclass[12pt]{article}

\usepackage{graphicx}
\usepackage{paralist}
\usepackage{hyperref}
\hypersetup{
    colorlinks=true,
    linkcolor=blue,
    urlcolor=cyan,
}
\usepackage{xspace}
\usepackage{amsfonts}
\usepackage{amsmath}

\newcommand{\latex}{\LaTeX\xspace}

\oddsidemargin 0mm
\evensidemargin 0mm
\textwidth 160mm
\textheight 200mm
\renewcommand\baselinestretch{1.0}

\pagestyle {plain}
\pagenumbering{arabic}

\newcounter{stepnum}

\title{
    SFWRENG 3XA3 Project Approval\\
    \large Department of Computing and Software\\
}
\author{}
\date{\today}

\begin {document}

\maketitle

\section{Team Name and Members}

\begin{tabular}{l l l}
    Team Name: & ScrumBot\\
    Group Number: & 304 \\
    \\
    Arkin Modi & modia1 & 400142497 \\
    Leon So & sol4 & 400127468 \\
    Timothy Choy & choyt2 & 400135272
\end{tabular}

\section{Original Project Information}
\subsection{Project Description}
The original source of our project came from Code-Plus-Plus' Discord Meeting Bot. It is an open-source Discord bot with some simple commands to run and plan meetings through Discord. The program is written in JavaScript, and has the commands to add and display meetings and welcome new users. The bot schedules meetings to a date and time.\\
A Discord bot is a specialized chat bot that runs on the Discord application. Discord is a voice and text chat application targeted at video game communities, but recently it has branched out for groups outside of just video games. There can be many uses for a Discord bot, from playing music to creating and planning meetings, which is what we plan on creating.

\subsection{Existing Tests}
Since the repository containing this meeting bot is very small, and has very few functions, we were unable to find any existing tests for the given project. However, there are some potential test cases which can be conducted, such as directly testing the existing Discord bot.

\subsection{URL of Original Project}
You can find the link to the original project \href{https://github.com/Code-Plus-Plus/discord-meeting-bot}{here}.

\section{Software Purpose and Scope}
Our plan is to build upon this open-source Discord bot, by putting a focus on software development and project management through the use of Trello and Jira connectivity. We plan on implementing features for managing projects developed using the agile development into meeting schedules and into the direct communication channels (i.e. Discord) which would greatly benefit software developers. This will allow better organization of Scrum meetings (i.e. retrospectives and stand-ups) and improve efficiency within the communication channel. Other components we plan on adding to the bot would be connections to Google API sources, such as allowing meetings times to be added straight to their Google Calendar, or using Google Maps to locate exactly where the meeting will be (if it were to be in person).

\section{Specialized Hardware Requirements}
This project requires no additional specialized hardware.

\section{Licenses}
The project requires the MIT License, which allows for copying, modifying, merging, publishing, distributing without restriction. Therefore, we should be able to use this open-source code as the basis for our project.

\section{Programming Language}
The primary programming language used to develop this project will by Python. Relevant Python libraries and APIs will also be used in the redevelopment of the Discord bot.

\section{Domain Knowledge}
The required knowledge to build a Discord bot is not too difficult. Most information about making such a bot could easily be found on the internet, through Google or looking at other Discord bots. The most difficult part of this project will be adding the components to the Discord bot, as well as understanding any APIs or libraries to be used in the redevelopment of the Discord bot, which is definitely feasible for us to complete in one term.

\end {document}
