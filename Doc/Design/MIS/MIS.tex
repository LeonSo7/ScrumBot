\documentclass[12pt, titlepage]{article}

\usepackage{fullpage}
\usepackage[round]{natbib}
\usepackage{multirow}
\usepackage{booktabs}
\usepackage{tabularx}
\usepackage{hyperref}
\usepackage{graphicx}
\usepackage{float}
\usepackage{amssymb}
\usepackage{amsmath}
\usepackage{hhline}
\hypersetup{
    colorlinks,
    citecolor=black,
    filecolor=black,
    linkcolor=red,
    urlcolor=blue
}
\usepackage[round]{natbib}

\newcounter{acnum}
\newcommand{\actheacnum}{AC\theacnum}
\newcommand{\acref}[1]{AC\ref{#1}}

\newcounter{ucnum}
\newcommand{\uctheucnum}{UC\theucnum}
\newcommand{\uref}[1]{UC\ref{#1}}

\newcounter{mnum}
\newcommand{\mthemnum}{M\themnum}
\newcommand{\mref}[1]{M\ref{#1}}

\title{SE 3XA3: Module Interface Specification\\ScrumBot}

\author{
    Team 304, ScrumBot
        \\ Arkin Modi, modia1
        \\ Leon So, sol4
        \\ Timothy Choy, choyt2
}
\date{Last Updated: \today}

\begin{document}

\maketitle

\begin{table}[H]
    \caption{Revision History} \label{TblRevisionHistory}
    \begin{tabularx}{\textwidth}{llX}
        \toprule
            \textbf{Date} & \textbf{Developer(s)} & \textbf{Change}\\
        \midrule
            March 9, 2020 & Timothy Choy & Copy template, ScrumBot Module\\
            March 10, 2020 & Leon So & MeetingList module\\
            March 11, 2020 & Leon So & Meeting module, MeetingTypes Module\\
            March 11, 2020 & Leon So & Project module, ProjectList Module, Meeting module, MeetingTypes Module, Generic Dict Module\\
        \bottomrule
    \end{tabularx}
\end{table}

\newpage

%% MEETING TYPES
\section* {MeetingTypes Module}

\subsection* {Module}
MeetingTypes

\subsection*{Uses}
N/A

\subsection*{Syntax}
\subsubsection*{Exported Constants}
N/A
\subsubsection*{Exported Types}
MeetingT = \{Grooming, StandUp, Retrospective, SprintPlanning\}
\subsubsection*{Exported Access Programs}
None

\subsection*{Semantics}
\subsubsection* {State Variables}
None
\subsubsection* {State Invariant}
None

\newpage

%% GENERIC LIST MODULE
\section*{Generic Dictionary Module}
\subsection*{Generic Template Module}
Dict(T)

\subsection*{Uses}
N/A

\subsection*{Syntax}
\subsubsection*{Exported Types}
Dict = ?
\subsubsection*{Exported Constants}
None
\subsubsection* {Exported Access Programs}
\begin{tabular}{|p{3cm}|p{3cm}|p{3cm}|p{4.5cm}|}
    \hline
    \textbf{Routine Name} & \textbf{In} & \textbf{Out} & \textbf{Exceptions} \\
    \hline
    new Dict & & Dict &\\
    \hline
    add & T & &\\
    \hline
    remove & $\mathbb{N}$ & & KeyError\\
    \hline
    toSeq & & set of ($\mathbb{N}$,T) & \\
    \hline
\end{tabular}

\subsection*{Semantics}
\subsubsection*{State Variables}
D: set of ($\mathbb{N}$,T) \\
c: $\mathbb{N}$
\subsubsection*{State Invariant}
$|D| \geq 0$\\
$c \geq 0$

\subsection*{Assumptions \& Design Decision}
\begin{itemize}
  \item The Dict(T) constructor is called for each object instance before any other access routine is called for that object.
\end{itemize}

\subsection*{Access Routine Semantics}
\noindent new Dict():
\begin{itemize}
    \item transition: $D, c := \langle \rangle, 0$
    
    \item output: $out := self$
\end{itemize}

\noindent add(e):
\begin{itemize}
    \item transition: $D :=  D\ ||\ \langle c,\ e \rangle, c := c + 1$
\end{itemize}

\noindent remove(id):
\begin{itemize}
    \item transition: $D :=  D - \langle id, e \rangle$
    \item exception: $exc := \langle id, e \rangle \notin D \Rightarrow \text{KeyError}$
\end{itemize}

\noindent toSeq():
\begin{itemize}
    \item output: $D$
\end{itemize}

\newpage

%% MEETING TYPES
\section* {MeetingTypes Module}

\subsection* {Module}
MeetingTypes

\subsection*{Uses}
N/A

\subsection*{Syntax}
\subsubsection*{Exported Constants}
N/A

\subsubsection*{Exported Types}
MeetingT = \{Grooming, StandUp, Retrospective, SprintPlanning\}

\subsubsection*{Exported Access Programs}
None

\subsection*{Semantics}
\subsubsection* {State Variables}
None

\subsubsection* {State Invariant}
None

\newpage

%% GENERIC LIST MODULE
\section*{Generic Dictionary Module}

\subsection*{Generic Template Module}
Dict(T)

\subsection*{Uses}
N/A

\subsection*{Syntax}
\subsubsection*{Exported Types}
Dict = ?

\subsubsection*{Exported Constants}
None

\subsubsection* {Exported Access Programs}
\begin{tabular}{|p{3cm}|p{3cm}|p{3cm}|p{4.5cm}|}
    \hline
    \textbf{Routine Name} & \textbf{In} & \textbf{Out} & \textbf{Exceptions} \\
    \hline
    new Dict(T) & & Dict(T) &\\
    \hline
    add & T & &\\
    \hline
    remove & $\mathbb{N}$ & & KeyError\\
    \hline
    to\_seq & & seq of ($\mathbb{N}$,T) & \\
    \hline
\end{tabular}

\subsection*{Semantics}
\subsubsection*{State Variables}
d: seq of ($\mathbb{N}$,T) \\
c: $\mathbb{N}$

\subsubsection*{State Invariant}
$|d| \geq 0$\\
$c \geq 0$

\subsection*{Assumptions \& Design Decisions}
\begin{itemize}
  \item The Dict(T) constructor is called for each object instance before any other access routine is called for that object.
\end{itemize}

\subsection*{Access Routine Semantics}
\noindent new Dict():
\begin{itemize}
    \item transition: $d, c := \langle \rangle, 0$
    
    \item output: $out :=$ self
\end{itemize}

\noindent add($e$):
\begin{itemize}
    \item transition: $d :=  d\ ||\ \langle c,\ e \rangle, c := c + 1$
\end{itemize}

\noindent remove($id$):
\begin{itemize}
    \item transition: $d :=  d - \langle id, e \rangle$
    \item exception: $exc := \langle id, e \rangle \notin d \Rightarrow \text{KeyError}$
\end{itemize}

\noindent to\_seq():
\begin{itemize}
    \item output: $out := d$ such that $(\forall i \in \mathbb{N} \mid 0 \leq i < |d| - 1 \cdot d[i] \leq d[i + 1] )$
\end{itemize}

\newpage
%% MEETING LIST
\section* {MeetingList Module}

\subsection*{Template Module}
MeetingList is a Dict(Meeting)

\newpage

%% Meeting
\section* {Meeting Module}

\subsection*{Module}
Meeting

\subsection* {Uses}
MeetingTypes

\subsection* {Syntax}
\subsubsection* {Exported Constants}
None

\subsubsection* {Exported Types}
Meeting = ?

\subsubsection* {Exported Access Programs}
\begin{tabular}{|l|l|l|l|}
    \hline
    \textbf{Routine Name} & \textbf{In} & \textbf{Out} & \textbf{Exceptions} \\
    \hline
    new Meeting & String, Date, Time, MeetingT & Meeting & \\
    \hline
    new Meeting & String, Date, Time, MeetingT, String & Meeting & \\
    \hline
    get\_name & & String &\\
    \hline
    get\_date & & Date &\\
    \hline
    get\_time & & Time &\\
    \hline
    get\_type & & MeetingT &\\
    \hline
    get\_description & & String &\\
    \hline
    set\_description & String & & \\
    \hline
\end{tabular}

\subsection* {Semantics}
\subsubsection* {State Variables}
$name$: String\\
$date$: Date\\
$time$: Time\\
$type$: MeetingT\\
$desc$: String

\subsubsection* {State Invariant}
None

\subsubsection* {Assumptions}
\begin{itemize}
  \item The Meeting constructor is called for each object instance before any other access routine is called for that object.
\end{itemize}


\subsubsection* {Access Routine Semantics}
\noindent new Meeting($n$, $d$, $t$, $type$)
\begin{itemize}
    \item transition: $name,\ time,\ date,\ type,\ desc := n,\ d,\ t,\ type,\ \text{None}$
    \item output: $out :=$ self
\end{itemize}

\noindent new Meeting($n$, $d$, $t$, $type$, $desc$)
\begin{itemize}
    \item transition: $name,\ time,\ date,\ type,\ desc := n,\ d,\ t,\ type,\ desc$
    \item output: $out :=$ self
\end{itemize}

\noindent get\_name()
\begin{itemize}
    \item output: $out := name$
\end{itemize}

\noindent get\_date()
\begin{itemize}
    \item output: $out := date$
\end{itemize}

\noindent get\_time()
\begin{itemize}
    \item output: $out := time$
\end{itemize}

\noindent get\_type()
\begin{itemize}
    \item output: $out := type$
\end{itemize}

\noindent get\_description()
\begin{itemize}
    \item output: $out := desc$
\end{itemize}

\noindent set\_description($s$)
\begin{itemize}
    \item transition: $desc := s$
\end{itemize}

\newpage

\section*{Task List Module}

\subsection*{Template Module}
TaskList is a Dict(Task)

\newpage

\section*{Task Module}

\subsection*{Module}
Task

\subsection*{Uses}
None

\subsection* {Syntax}
\subsubsection* {Exported Constants}
None

\subsubsection* {Exported Types}
Task = ?

\subsubsection* {Exported Access Programs}
\begin{tabular}{|l|l|l|l|}
    \hline
    \textbf{Routine Name} & \textbf{In} & \textbf{Out} & \textbf{Exceptions} \\
    \hline
    new Task & String, Date, Time & Task & \\
    \hline
    new Task & String, Date, Time, Details & Task & \\
    \hline
    get\_deadline & & (Date, Time) & \\
    \hline
    get\_details & & String & \\
    \hline
    get\_feedback & & seq of String& \\
    \hline
    add\_feedback & String & & \\
    \hline
    rm\_feedback & $\mathbb{N}$ &&\\
    \hline
    set\_details & String & & \\
    \hline
\end{tabular}

\subsection*{Semantics}
\subsubsection*{State Variables}
$name$: String\\
$deadline$: (Date, Time) \\
$details$: String\\
$feedback$: seq of String

\subsubsection*{State Invariant}
None

\subsubsection*{Assumptions}
\begin{itemize}
  \item The Task constructor is called for each object instance before any other access routine is called for that object.
\end{itemize}

\subsubsection*{Access Routine Semantics}
\noindent new Task($s, d, t$)
\begin{itemize}
    \item transition: $name, deadline, details := s, (d, t), \text{None}$
    \item output: $out :=$ self
\end{itemize}

\noindent new Task($s, d, t, details$)
\begin{itemize}
    \item transition: $name, deadline, details := s, (d, t), details$
    \item output: $out :=$ self
\end{itemize}

\noindent get\_deadline()
\begin{itemize}
    \item output: $out := deadline$
\end{itemize}

\noindent get\_details()
\begin{itemize}
    \item output: $out := (details = \text{None} \Rightarrow ``\text{No details}"\ |\ details)$
\end{itemize}

\noindent get\_feedback()
\begin{itemize}
    \item output: $out := feedback$
\end{itemize}

\noindent add\_feedback($s$)
\begin{itemize}
    \item transition: $feedback := feedback\ ||\ s$
\end{itemize}
\newpage

\noindent rm\_feedback($s$)
\begin{itemize}
    \item transition: $feedback := feedback - s$
\end{itemize}

\noindent set\_details($s$)
\begin{itemize}
    \item transition: $details := s$
\end{itemize}

\newpage

%% Sprint
\section* {Sprint Module}

\subsection*{Module}
Sprint

\subsection* {Uses}
TaskList, Task 

\subsection* {Syntax}
\subsubsection* {Exported Constants}
None

\subsubsection* {Exported Types}
Sprint = ?

\subsubsection* {Exported Access Programs}
\begin{tabular}{|l|l|l|l|}
    \hline
    \textbf{Routine Name} & \textbf{In} & \textbf{Out} & \textbf{Exceptions} \\
    \hline
    new Sprint & $\mathbb{N}$ & Sprint & \\
    \hline
    get\_tasks & & seq of Task &\\
    \hline
    add\_task & Task & & \\
    \hline
    rm\_task & $\mathbb{N}$ & &\\
    \hline
\end{tabular}

\subsection*{Semantics}
\subsubsection*{State Variables}
$sprint\_num$: $\mathbb{N}$\\
$tasks$: TaskList
\subsubsection*{State Invariant}
None

\subsubsection*{Assumptions}
\begin{itemize}
  \item The Sprint constructor is called for each object instance before any other access routine is called for that object.
\end{itemize}

\subsubsection*{Access Routine Semantics}
\noindent new Sprint($n$)
\begin{itemize}
    \item transition: $sprint\_num, tasks := n, TaskList()$
    \item output: $out :=$ self
\end{itemize}

\noindent get\_task($n$)
\begin{itemize}
    \item output: $out := tasks.to\_seq()$
\end{itemize}

\noindent add\_task($task$)
\begin{itemize}
    \item transition: $tasks := tasks.add(task)$
\end{itemize}

\noindent rm\_task($n$)
\begin{itemize}
    \item transition: $tasks := tasks.renove(n)$
\end{itemize}

\newpage

%% Project List Module
\section* {ProjectList Module}

\subsection*{Template Module}
ProjectList is a Dict(Project)

\newpage

%% Project
\section* {Project Module}

\subsection*{Module}
Project

\subsection* {Uses}
Sprint, Tasklist, Task

\subsection* {Syntax}
\subsubsection* {Exported Constants}
None

\subsubsection* {Exported Types}
Project = ?

\subsubsection* {Exported Access Programs}
\begin{tabular}{|l|l|l|l|}
    \hline
    \textbf{Routine Name} & \textbf{In} & \textbf{Out} & \textbf{Exceptions} \\
    \hline
    new Project & String & Project & \\
    \hline
    new Project & String, String & Project & \\
    \hline
    get\_desc & & String &\\
    \hline
    get\_meetings & & seq of Meeting &\\
    \hline
    get\_rqes & & seq of String & \\
    \hline
    get\_sprints & & seq of Sprint & \\
    \hline
    set\_desc & String & &\\
    \hline
    add\_meeting & Meeting & & \\
    \hline
    add\_rqe & String & & \\
    \hline
    pop\_sprint & $\mathbb{N}$ & & IndexError\\
    \hline
    push\_sprint & Sprint & & \\
    \hline
    rm\_meeting & $\mathbb{N}$ &  &\\
    \hline
    rm\_rqe & $\mathbb{N}$ &  & IndexError\\
    \hline
\end{tabular}

\subsection*{Semantics}
\subsubsection*{State Variables}
$name$: String\\
$desc$: String\\
$meetings$: MeetingList\\
$rqes$: seq of String\\
$sprints$: seq of Sprint

\subsubsection*{State Invariant}
None

\subsubsection*{Assumptions}
\begin{itemize}
  \item The Project constructor is called for each object instance before any other access routine is called for that object.
\end{itemize}

\subsubsection*{Access Routine Semantics}
\noindent new Project($n$)
\begin{itemize}
    \item transition: $name, desc, rqes, sprints := n,\ \text{None},\ [],\ []$
    \item output: $out :=$ self
\end{itemize}

\noindent new Project($n$, $d$)
\begin{itemize}
    \item transition: $name, desc, rqes, sprints := n,\ d,\ [],\ []$
    \item output: $out :=$ self
\end{itemize}

\noindent get\_desc()
\begin{itemize}
    \item output: $out := (desc = \text{None} \Rightarrow ``\text{No description}"\ |\ desc)$
\end{itemize}

\noindent get\_meetings()
\begin{itemize}
    \item output: $out := meetings.to_seq()$
\end{itemize}

\noindent get\_rqes()
\begin{itemize}
    \item output: $out := rqes$
\end{itemize}

\noindent get\_sprints()
\begin{itemize}
    \item output: $out := sprints$
\end{itemize}

\noindent set\_desc($s$)
\begin{itemize}
    \item transition: $desc := s$
\end{itemize}

\noindent add\_meeting($meeting$)
\begin{itemize}
    \item transition: $meetings := meetings.add(meeting)$
\end{itemize}

\noindent add\_rqe($s$)
\begin{itemize}
    \item transition: $rqes := rqes\ ||\ s$
\end{itemize}

\noindent pop\_sprint()
\begin{itemize}
    \item transition: $sprints := sprints[0: |sprints| - 2]$
    \item exception: $exc := |sprints| = 0 \Rightarrow \text{IndexError}$
\end{itemize}

\noindent push\_sprint($sprint$)
\begin{itemize}
    \item transition: $sprints := sprints\ ||\ sprint$
\end{itemize}

\noindent rm\_meeting($n$)
\begin{itemize}
    \item transition: $meetings := meetings.remove(n)$
\end{itemize}

\noindent rm\_rqe($n$)
\begin{itemize}
    \item transition: $rqes := rqes - rqes[n]$
\end{itemize}

\end{document}
